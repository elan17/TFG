\documentclass{tfg_domingo}
% \documentclass[numeros]{tfg_domingo}

\autor{Juan Toca Mateo}
\titulo{Secuencias binarias y sus aplicaciones}
% Título corto para los encabezamientos de pagina:
\corto{Binary sequences and their applications} % En blanco si no es necesario recortarlo.
\ingles{Binary sequences and their applications}
\fecha{septiembre de 2020}
% La normativa prescribe «cuatro o cinco palabras clave, en
% español y en inglés, para su indexación en el repositorio
% de TFG».
\palabras{secuencias binarias, baja autocorrelación, búsqueda exhaustiva, ramificación y poda}%
  {binary sequences, low autocorrelation, exhaustive search, branch-and-bound}

\usepackage{lipsum} % Esto solo es relleno.
\newcommand{\domingo}[1]{\textcolor{red}{#1}}
\begin{document}

% Si alguna palabra se divide entre dos líneas en un punto
% indebido, podemos indicar aquí los puntos de corte
% aceptables (si los hay), p. ej,
% \hyphenation{ba-rro-co, frío, cria-do, su-per-ra-tón}
\hyphenation{Dijkstra new-speak}

\portada
\frontmatter
% \sucinto{A Sofía}
\gracias{\input{Chapters/agradecimientos.txt}}
\resumen{Generar familias de secuencias con correlación acotada entre otras
propiedades es de interés en diversas áreas como criptografía, comunicaciones
inalámbricas y marcas de agua digitales. Aplicaciones comerciales como el GPS,
servicios de localización como «Ultra Wide Band» (UWB) o usos militares como el
Radar se han desarrollado y mejorado  a partir de la búsqueda de estas
secuencias mediante el estudio de su función tanto de autocorrelación como la
correlación entre los miembros de una misma familia.\\

En este proyecto, nos centramos en una técnica para la
construcción algebraica de secuencias donde, a traves de una secuencia de
desplazamientos y una secuencia con buenas propiedades, se genera una nueva
secuencia. El objetivo es la búsqueda exhaustiva de secuencias de mayor
longitud que las ya existentes en la literatura. Para este fin, se ha
desarrollado un software de apoyo al diseñador con capacidad de ser desplegado
en un nodo de supercomputación para asistir a la búsqueda de dichas secuencias
y la comprobación de sus propiedades. La finalidad de estas secuencias, entre
otros posibles usos, son radares y sistemas de localización con mayor resolución
espacial y temporal.\\
}{The analisis of the correlation of signals it's a key piece in several
engineering developments such as GPS, sonar or the correction of errors at
information transmission. Having shaped so many diverse activities like
driving, cartography or internet usage, the study of the correlation function
may lead to new developments or improvements in the ones that already exist.\\

In this project, we focus on the generation of new pseudorandom sequences for
more accurate sonar or GPS technologies. To do so, a new software deployable in
a new supercomputer has been developed to assist in the search of those
sequences.
}
\tableofcontents

\mainmatter


\chapter{Introduction}

There is no doubt that signals have changed drastically the way we live.
Physical maps have passed out long ago in favor of real-time position tracking
systems. Who needs a meter when you can send a signal to measure distances
accuratly? Even publishing this project it's thanks to the hard work of
engineers that squeeze the capabilities of the carrier wave to transport
information throughout Internet. \\

In this chapter, we will introduce the correlation, autocorrelation and
crosscorrelation functions and it's mathematical properties. We will also take
a look at pseudorandom noise, it's properties and practical applications.

\section{Binary sequences}

Before diving into the matter, it is needed to point out the specific definition
of binary sequences for this field:\\

In computer science, binary symbols are normally defined with the set ${0, 1}$ to
be used in boolean logic and arithmetics. This approach is useful when dealing
with the inner workings of a computer or its thworetical constructs.\\

However, the discussed topic in this work takes a mathematical approach. In fact,
most parts of the theory exposed here are just specific versions of general
definitions for complex numbers. With this in mind and to avoid redefining
needlessly the equations, we will stick to sequences composed by the elements
$1$ and $-1$, taken from the ring of integers.

\section{Correlation function}

According to \citet{golomb_ref}, the correlation function measures how similar
two phenomena are. If properly normalized, the function ranges from
+1 (identical) to -1 (opposite); 0 meaning completly unrelated phenomena.
If those phenomena are represented as vectors, the correlation can be conceived
as the normalized dot product between those 2 vectors.
In the discrete case where both sequences have the same length, the one in which
this project focus on, the normalized version is defined as follows:

\begin{definition}[Normalized correlation]\label{def:1}

Given $\alpha$ and $\beta$ two vectors of the same length n and $\alpha_{i}$
and $\beta_{i}$ the components of the vectors:

\begin{equation}\label{eq:1}
C(\alpha , \beta)=\frac{(\alpha \cdot  \beta)}{|\alpha||\beta|}=\frac{\sum_{i=1}^{n} \alpha_{i}\beta_{i}}{(\sum_{i=1}^{n} \alpha_{i}^{2})^{\frac{1}{2}}(\sum_{i=1}^{n} \beta_{i}^{2})^\frac{1}{2}}.
\end{equation}
\end{definition}

Notice that in this vector
representation:
\begin{itemize}
  \item Orthogonal vectors have a correlation value of 0 .
  \item Vectors with the same direction and orientation have a correlation
  value of 1.
  \item Vectors with the same direction but opposite orientation have a
  correlation value of -1.
\end{itemize}

Even though the normalized version is a good way to grasp the
concept of the degree of similarity between two phenomen, for the rest of the
document the unnormalized version is going to be used unless it is stated.
This definition of the correlation function have several advantages for our
research as it is simpler and carries the same amount of information while
saving us some computation resources and complexity on our theoretical
analysis. The unnormalized correlation is defined as:

\begin{definition}[Unnormalized correlation]\label{def:2}
  Given $\alpha$ and $\beta$ two vectors of the same length n and $\alpha_{i}$
  and $\beta_{i}$ the components of the vectors:
  \begin{equation}\label{eq:2}
    C(\alpha , \beta) = (\alpha \cdot  \beta) = \sum_{i=1}^n(\alpha \odot \beta)_{i}= \sum_{i=1}^{n} \alpha_{i}\beta_{i}
  \end{equation}

  where "$\odot$" represents the pointwise product of vectors.

\end{definition}

%Adapting this function from vectors to finite digital signals is straight
%forward, as they can be defined in terms of a vector that lives in a vector
%space of dimension equal to the length of the signal.

\begin{figure}[ht!] % [h!] fuerza que el elemento se sitúe
                    % en la posición señalada, en vez de al
                    % comienzo de una página.
\begin{center}
\includegraphics[width=0.7\linewidth]{Chapters/Introduction/signals_correlation}
\end{center}
\caption{A graphical representation of two vectors an their pointwise product
with an unnormalized correlation between them of -2 (-1 -1 -1 +1 -1 +1)}
\label{introduction_signals_hadamard}
\end{figure}

As represented in Figure \ref{introduction_signals_hadamard}, unnormalized
correlation can be performed using only integer arithmetics, multiplication
and addition, thus becoming easier to implement using the resources  avaliable
on a digital device.










\subsection{Autocorrelation function}

Going on with the lecture of \citet{golomb_ref}, the autocorrelation function
is a measure of how the correlation behaves if, for a given sequence, a
circular shift is applied and then correlated with the original sequence for every
possible shift. It is defined for periodic sequences as follows:

\begin{definition}[Autocorrelation]\label{def:3}

Given the function C defined in Equation \eqref{eq:2} and n the length of the
sequence S

\begin{equation}\label{eq:3}
  shift(S, \tau)_i = S_{(i+\tau) \bmod n}
\end{equation}
\begin{equation}\label{eq:4}
  A(S)_{\tau} = C(S, shift(S, \tau)) = \sum_{i=1}^{n}S_{i}S_{(i+\tau) \bmod n}
\end{equation}

\end{definition}

\begin{figure}[ht!] % [h!] fuerza que el elemento se sitúe
                    % en la posición señalada, en vez de al
                    % comienzo de una página.
\begin{center}
\includegraphics[width=0.7\linewidth]{Chapters/Introduction/signals_autocorrelation}
\end{center}
\caption{A graphical representation of the autocorrelation of a sequence with  a shifted version of itself.}
\label{introduction_signals_autocorrelation}
\end{figure}

An example  is shown in Figure
\ref{introduction_signals_autocorrelation} in which some important properties of
the autocorrelation of sequences can be observed:

\begin{theorem}\label{theorem:1.2.1}
  Given a sequence S, the autocorrelation value for $\tau = 1$ is:
    \begin{equation}
      A(S)_{1}=C(S, S)=\sum_{i=1}^{n}S_{i}^2
    \end{equation}
\end{theorem}

\begin{corollary}
  Given the unnormalized autocorrelation of a sequence, it can be normalized
  by dividing it as follows:
  \begin{equation}
    A'(S)_{\tau} = \frac{A(S)_{\tau}}{A(S)_{1}}
  \end{equation}
\end{corollary}

\begin{proof}
  Using Equations \eqref{eq:1} and \eqref{eq:4}, Equation \eqref{eq:4} can be
  normalized as follows:

    $$A'(S)_{\tau} = C'(S, shift(S, \tau)) = \frac{C(S, shift(S, \tau))}{(\sum_{i=1}^{n} S_{i}^{2})^{\frac{1}{2}}(\sum_{i=1}^{n} S_{i+\tau}^{2})^\frac{1}{2}} = \frac{A(S)_{\tau}}{\sum_{i=1}^{n} S_{i}^{2}} = \frac{A(S)_{\tau}}{A(S)_{1}}$$

  Keep in mind that, even though $S_{i}^2$ and $S_{i+\tau}^2$ aren't the same
  element, the elements of the shifted version are the same as the original
  sequence so the total sum is the same.
\end{proof}

\begin{corollary}\label{autocorrelation:coro:1}
  Given the autocorrelation of a sequence, $A(S)_{1}$ will always be the
  maximum value of the autocorrelation.
\end{corollary}

\begin{property}

  If the components of the original sequence form a field, the components of
  its autocorrelation belong to the same field.

\end{property}

Even though this seems a naive property, this will prove
useful when we introduce the algorithm based in the Fourier Transform to
compute the autocorrelation function.









\subsection{Crosscorrelation function}

The crosscorrelation function measures how a sequence correlates with all
the posible shifts of another sequence. This function is useful to analyze if two
signals can be mistaken one for another by a receiver when delays in time occur.


\begin{definition}[Crosscorrelation]\label{def:4}
  Given C the correlation function defined in Equation \eqref{eq:2}, shift as the function defined in Equation \eqref{eq:3} and n the length of both sequences:
  \begin{equation}\label{eq:7}
    CC(S1, S2)_{\tau} = C(S1, shift(S2, \tau)) = \sum_{i=1}^{n}S1_{i}S2_{(i+\tau) \bmod n}
  \end{equation}
\end{definition}

\begin{figure}[ht!] % [h!] fuerza que el elemento se sitúe
                    % en la posición señalada, en vez de al
                    % comienzo de una página.
\begin{center}
\includegraphics[width=0.7\linewidth]{Chapters/Introduction/signals_crosscorrelation}
\end{center}
\caption{A graphical representation of the crosscorrelation between two sequences.}
\label{introduction_signals_crosscorrelation}
\end{figure}

\begin{definition}\label{lem:1}
  Given a sequence S, the crosscorrelation function  (CC) can be defined in Equation
  \eqref{eq:7} with the autocorrelation function A defined in Equation \eqref{eq:4}:
  \begin{equation}\label{eq:8}
    CC(S, S) = A(S)
  \end{equation}
\end{definition}












\section{Pseudorandom noise (PN)}

Noise have a different meaning depending on the field of study in which is
used. In our case we are going to work with random vectors,
which are defined as vectors whose components are realizations of independent
and uniform distributed variables\cite{white_noise}.\\

Even though noise in general is usually seen as an unwanted phenomena that
limits the amount of information that can be transmitted through a
channel\cite{shannon_noise}, it is useful to study its properties, such as

%This practical applications exploit an important noise property:

\begin{property}
 The expected value of the autocorrelation of a random vector is zero for
 every component
  where $\tau \neq 0$ \cite{everett}.
\end{property}

Taking a radar as an example, using this property the distance can be computed
just by sending a random sequence in the direction of a target and start
correlating the received signal with the original one. As the autocorrelation
of random vector is different from zero only when the shift is zero, the peak
in the autocorrelation gives in which time instant the signal has returned to
the receiver. With that time instant, the round-trip time can be computed and then
the actual distance using the propagation speed of the wave.\\

In the case of GPS, the restrictions imposed to the random sequence are
stronger. First of all, as several signals will be transmitted in the same
frequency, a set of sequences with good auto and crosscorrelation
properties between them is needed. In other words, the maximum crosscorrelation
function between two given sequences must trend to zero in every component, except
when $\tau = 0$ and both sequences are the same.\\

Random sequences do not guarantee good properties of correlation, even noise
measured from natural phenomena can generate sequences with poor correlation
properties. As stated before, important technologies depend on sequences with
these properties, therefore, it is necessary to develop methods that are
efficient to create sequences with properties similar to those of random in a
deterministic and efficient fashion.\\

This kind of sequences are called Pseudo Noise (PN). Although for most
applications, the off peak autocorrelation and crosscorrelation should be equal to
zero, it is conjetured that such sequence do not exist apart from length four. That
is why pseudo noise are use in practical applications. Then a threshold is defined
so that the system will not mistake intermediate values with the peak in the
autocorrelation.


\begin{figure}[ht!] % [h!] fuerza que el elemento se sitúe
                    % en la posición señalada, en vez de al
                    % comienzo de una página.
\begin{center}
\includegraphics[width=0.7\linewidth]{Chapters/Introduction/signals_prn}
\end{center}
\caption{A pseudorandom noise sequence an it's autocorrelation function.
Notice that this pseudonoise sequence isn't perfect noise.}
\label{introduction_signals_autocorrelation}
\end{figure}

% TODO Define family of sequences

% TODO Define flat autocorrelation



\chapter{Pseudonoise generation}

As explained in the previous chapter, PRN codes are useful in tecnologies
that need properties similar to those of white noise. In this chapter, we
are going to introduce some state-of-the-art tecniques in PRN generation.

\section{Maximum Length Sequence(MLS)}

MLS are an exponential binary PRN construction that was initially concived
using linear feedback shift registers(LFSR). The particular type of LFSR used
in MLS can be simulated with extensions of binary Finite Fields. The
definitions are:

\begin{definition}[LFSR]
  A MLS is a binary sequence generated by an LFSR that, given an initial state
  different from 0, it cycles between all posible states except 0.
\end{definition}

Which, as shown in \citet{golomb_ref}, is equivalent to:

\begin{definition}[Finite Fields]
  Given $E/GF(2)$, $\alpha$ a primitive element of $E$ and $S$ the resulting
  sequence:
  \begin{equation}
    S_{i} = trace(\alpha^{i})
  \end{equation}
\end{definition}

\begin{property}
  A MLS will always be of length of the form $2^{n}-1$ where n is an arbitrary
  natural number.
\end{property}

\begin{property}
  A MLS sequence will always have an autocorrelation function such as all the
  components will be -1 except when $\tau = 0$
\end{property}

\begin{figure}[ht!]
  \inputpython{Chapters/PRN_generation/example_mls.py}{0}{100}
  \caption{An example of a posible implementation of MLS sequences}
  \label{mls:fig:1}
\end{figure}

Notice that, even though the construction is exponential, the complexity of
the algorithm is $O(n)$ when $n$ is the size of the sequence. The problem is
that we might need sequences of arbitrary size in some applications. As we will
see in a following chapter, the complexity of computing the autocorrelation
with the fourier transform approach is $O(n*log(n))$ so using longer sequences
than needed has a direct impact on the performance of the system. \\

This sequences by themselfs might not a be huge deal because they don't define
a way to build families of well crosscorrelated sequences but they are the
building blocks for other contructions such as the Gold Codes used in GPS and
CDMA.

\section{Gold Codes}

Gold codes\cite{gold_codes} are a family of sequences, derived from
m-sequences, with very important properties that are used in several
applications such as wireless communication and geolocalisation. A Gold Code
generator gets two m-sequences sequences that fulfill:

\begin{property}
  Given two m-sequences that can generate Gold Codes, $S1$ and $S2$, of length
  $2^{n}-1$, $CC$ the crosscorrelation function defined in Equation \eqref{eq:7}:
    \begin{equation}\label{gold:eq:1}
      max |CC(S1, S2)| \leq 2^{\frac{n+2}{2}}
    \end{equation}
\end{property}

And XORs all their relative shifts generating a family of $2^{n} + 1$ sequences
($2^{n} - 1$ XORed sequences + 2 m-sequences).

\begin{property}
  Given any two sequences, S1 and S2, from a Gold family of sequences of length
  $2^{n}-1$, $CC$ the crosscorrelation function defined in Equation \eqref{eq:7}:
  \begin{equation}
        max |CC(S1, S2)| \leq \left\{\begin{array}{lr}
            2^{\frac{n+2}{2}}+1 & \textnormal{ if n is even } \\
            2^{\frac{n+1}{2}}+1 & \textnormal{ if n is odd } \\
        \end{array}\right.
  \end{equation}
\end{property}

% TODO: Add a  code example

This property means that the crosscorrelation between any given pair of
sequences from a Gold family is low enough to differentiate them. This proves
useful when several devices are transmiting in the same frecuency and we must
treat signals that are not the one we want to receive as noise.\\

Gold also showed  a way to generate this pair of sequences,
using a decimation of one m-sequence.

\begin{definition}[Decimation]
  Given a sequence $S$ of length $n$, a decimation by $q$ of $S$ is defined as:
  \begin{equation}
    S[q]_{i} = S_{((q·i) \bmod n)}
  \end{equation}
\end{definition}

\begin{property}
  Given a m-sequence $S$ of length $2^n - 1$ where n is odd and and a
  coprime of $n$ named $k$, the sequence pair $(S, S[2^{k} + 1])$ number
  fulfills Equation \eqref{gold:eq:1}.
\end{property}

\begin{figure}[ht!]
  \inputpython{Chapters/PRN_generation/example_gold.py}{0}{100}
  \caption{An example implementation of a generation of a family of gold
  sequences relying in the example at Figure \ref{mls:fig:1}.}
  \label{}
\end{figure}

Notice that this construction has the same problem as m-sequences. It's
an exponential contruction so it might not be enough for some applications.

\section{Legendre sequences}

Legendre sequences, as explained in \citet{legendre_sequences}, are binary
sequences defined through quadratic residues. A quadratic residue modulo $p$ is an integer number $x$ between $0$ and $p-1$ such that an integer $y$ exists satisfying $x=y^2 \bmod p$.

\begin{definition}
  Let $p$ be an odd prime and the  Legendre Symbol function  be
    \begin{equation}
      LSy(n, p) = \left\{\begin{array}{lr}
          1  & \textnormal{if n is a quadratic residue mod p}   \\
          -1 & \textnormal{otherwise} \\
      \end{array}\right.
    \end{equation}
  The Legendre sequence can be defined as:
    \begin{equation}
      LSs(p)_{i} = LSy(i-1, p) \textnormal{  where  } 0 \leq i \leq p-1
    \end{equation}
\end{definition}

\begin{figure}[ht!]
  \inputpython{Chapters/PRN_generation/example_legendre.py}{0}{100}
  \caption{An example implementation of the generation of a Legendre sequence.}
  \label{}
\end{figure}
 Legendre sequences with length  satisfying $p=3\bmod 4$ have flat autocorrelation, i.e.
 the value of the autocorrelation for any nonzero shift is $-1$. There is a generalization of this property by \citet{legendre_sequences}  but it requires the sequence
 to be non-binary.\\

As $p$ is the length of the generated sequence, the distribution of Legendre
sequences is related to the Prime Number Theorem\cite{manfred_prime_theorem}. This
means that Legendre sequences have more possible lengths than in m-sequences or
other constructions based on them. However, Legendre sequences have the drawback
that there exists only one per sequence length.

\section{Composition method}

\subsection{Algorithm}

The composition method  was introduced by \citet{tirkel_composition}, initially called prime
arrays,  uses a base sequence and a sequence of shifts to create a matrix  whose columns are
shifts of the base sequence as follows:

\begin{definition}[Composite matrix]
  Given a base sequence $S$ of length $n$ and a sequence of integers $T$ of
  length $m$ such that:
  \begin{equation}\label{composition:eq:1}
    0 \leq T_{i} < n
  \end{equation}
  \begin{equation}\label{composition:eq:2}
    gcd(n, m) = 1
  \end{equation}

  Given the $shift$ function defined in Equation \eqref{eq:3},
   the composite matrix is defined as

  \begin{equation}\label{composition:eq:3}
    CM(S, T) = \begin{bmatrix}
      shift(S, T_{0})_{0} & shift(S, T_{1})_{0} & \dots & shift(S, T_{m-1})_{0} \\
      shift(S, T_{0})_{1} & shift(S, T_{1})_{1} \\
      \vdots & & \ddots \\
      shift(S, T_{0})_{n-1} & & & shift(S, T_{m-1})_{n-1}
    \end{bmatrix}
  \end{equation}

%  In other words, each column represents a shift of the base sequence defined
 % by the sequence of shifts.
\end{definition}

\begin{definition}[Composite sequence]
  Given a base sequence $S$ of length $n$, a shift sequence $T$ of
  length $m$ that satisfy Equations \eqref{composition:eq:1} and
  \eqref{composition:eq:2} and the composite matrix defined at Equation
  \eqref{composition:eq:3}, the composite sequence is defined as:
  \begin{equation}
    CS(S, T)_{i} = CM(S, T)_{(i \bmod m), (i \bmod n)}
  \end{equation}
\end{definition}
The fact that the sequence has length $nm$ and the correlation function can be deduce from the matrix is a consequence of the Chinese Remainder Theorem.  This is informally called the diagonal method, because it unfolds the array following the diagonal and using the wrap around provided by the modulo operation.
\begin{figure}[ht!]
  $$S = \begin{bmatrix}
    0 & 1 & 2 & 3 & 4\\
  \end{bmatrix}$$
  $$T = \begin{bmatrix}
    0 & 2 & 1 & 4 & 3 \\
  \end{bmatrix}
  $$
  $$CM(S, T) = \begin{bmatrix}
  0 & 3 & 4 & 1 & 2 & 4\\
  1 & 4 & 0 & 2 & 3 & 0\\
  2 & 0 & 1 & 3 & 4 & 1\\
  3 & 1 & 2 & 4 & 0 & 2\\
  4 & 2 & 3 & 0 & 1 & 3\\
  \end{bmatrix}
  $$
  $$CS(S, T) = \text{[0 4 1 4 1 4 1 0 2 0 2 0 2 1 3 1 3 1 3 2 4 2 4 2 4 3 0 3 0 3]}\\
  $$
  \caption{Example of a computation of the composition method. For clarity purposes, the example uses a non binary base sequence.}
  \label{}
\end{figure}

\begin{definition}[Composite matrix correlation]
  Given two composite matrices $M$ and $R$ with $n$ rows and $m$ columns,
  its correlation is defined as
  \begin{equation}
    C(M, R) = \sum_{i=0}^{m-1}\sum_{j=0}^{n-1}M_{j, i}R_{j,i}.
  \end{equation}
\end{definition}

\begin{definition}[Composite matrix shift]\label{definition:composition:shift}
  Given a composite matrix $M$ of $n$ rows and $m$ columns, the shift function
  is defined as:
  \begin{equation}
    shift(M, \tau)_{i, j} = M_{(i-\tau \bmod m), (j-\tau \bmod n)}
  \end{equation}
\end{definition}
This means that the following relation can be established:
  \begin{corollary} Getting a shift of the composite sequence is equivalent to
  applying a shift to the matrix and then extracting the corresponding
  sequence.
    \begin{equation}
      shift(CS(S, T), \tau)_{i} = shift(CM(S, T), \tau)_{(i \bmod m), (i \bmod n)}
    \end{equation}
  \end{corollary}

The previous corollary has an interesting implication for computing the autocorrelation function.

\begin{corollary}
  Given a composite matrix $M$ of $n$ rows and $m$ columns, its
  autocorrelation function is defined as:
  \begin{equation}
    \begin{split}
      A(M)_{\tau} = C(M, shift(M, \tau)) = \sum_{i=0}^{m-1}\sum_{j=0}^{n-1}M_{j, i}M_{(j-\tau \bmod m),(i - \tau \bmod n)} = \\
      \sum_{i=0}^{m-1}\sum_{j=0}^{n-1}shift(S, T_{i})_{j}shift(S, T_{(i - \tau \bmod n)})_{(j-\tau \bmod m)}.
    \end{split}
  \end{equation}
\end{corollary}

Notice that if $i$ is kept constant, a particular component of the
autocorrelation function of $S$ is obtained.

\begin{property} \label{composition:prop:1}
The autocorrelation function of the composite
sequence can be defined in terms of the autocorrelation function as follows:

\begin{equation}
  \sum_{j=0}^{n-1}shift(S, T_{i})_{j}shift(S, T_{(i - \tau \bmod n)})_{(j-\tau \bmod m)} = A(S)_{|((T_{j} - \tau) \bmod m) - T_{(j + \tau) \bmod n}|}
\end{equation}
\begin{equation}
  A(M)_{\tau} = \sum_{i=0}^{m-1} A(S)_{|((T_{j} - \tau) \bmod m) - T_{(j + \tau) \bmod n}|}
\end{equation}

\end{property}

This property is useful  as a fast method for computing the autocorrelation function for composed sequences and this will be implemented for efficiency.

\subsection{Costas arrays}

Costas arrays, discovered independently by John P. Costas\cite{costas_costas}
and E.N. Gilbert \cite{gilbert_costas} in 1965, are a family of sequences highly
used in radar and sonar applications. The equivalent definitions from both works has been included as both will be useful for different purposes.

\begin{definition}[Costas array by Costas \cite{costas_costas})]
 An Square matrix of size $n×n$  is filled with 0s and 1s such that there is a single
 1 in each row and column. Thus if the
  displacement vector is calculated by substracting all pairs of positions of the 1s, then  the results are unique.
\end{definition}

This definition is used in several deployments of sonar and radar to generate
systems with a good ambiguity function, in other words, tolerant to the Doppler
effect.

\begin{figure}[ht!]
  $$
  \begin{bmatrix}
   1&0&0&0\\
   0&0&0&1\\
   0&1&0&0\\
   0&0&1&0
  \end{bmatrix}
  $$
  \caption{An example of a Costas array}
  \label{fig:costas_1}
\end{figure}

Notice that the representation can be compacted by just having a list of the
rows in which each 1 lives:

\begin{figure}[ht!]
  $$[3, 1, 0, 2]$$
  \caption{The compact representation of the Costas array of Figure
  \ref{fig:costas_1}.}
  \label{fig:costas_2}
\end{figure}

This representation is equivalent to the definition of a Costas array provided
by Gilbert:

\begin{definition}[Distinct difference permutations]\label{def:costas_1}
  Given a sequence of integers $S$ of length $n$ such that:
    \begin{equation}\label{eq:costas_1}
      0 \leq S_{i} < n
    \end{equation}
  It is said that $S$ is a distinct difference permutation $r$ apart if,
  for any given pair $(S_{i}, S_{j})$, satisfies:
    \begin{equation}\label{eq:costas_2}
      S_{i} - S_{i+r} \not \equiv S_{j} - S_{j+r} \bmod n
    \end{equation}
\end{definition}

\begin{definition}[Costas array by Gilbert \cite{gilbert_costas}]
  Given a sequence $S$ satisfying Equation \eqref{eq:costas_1}, it is called a
  Costas array if, for any given $r$ value, it satisfies Equation
  \eqref{eq:costas_2}.
\end{definition}

This compact representation can be feeded into the composition method as
a sequence of shifts generating interesting new
sequences\cite{moreno_costas}.\\

Several construction methods for Costas arrays have
been proposed. For sake of simplicity, just the Welch
construction will be introduced following the work of  Gilbert\cite{gilbert_costas}.

Given a prime number $p$ and a primitive root $g$ of $p$ such as all the powers of $g$ in $[0, p-1]$ modulo $p$ are different from each other.  A Costas array $S$
can be constructed as follows:
\begin{equation}
  S_{i} \equiv g^{i} \bmod p, \quad 0 \leq i <p-1
\end{equation}

This  construction can generate sequences of  length $p-1$ for a given $g$. As the number of possible sequences depend on the number of primitive
elements of the finite field of order $p$, the number of possible Costas arrays
for a given length using this construction is $\phi(p-1)$ where $\phi$ is the
Euler's totient function(see
\citet{manfred_totient}).



\chapter{Exhaustive pseudonoise search}
  \section{Previous work}

  As shown in the previous chapter, arithmetic methods for finding sequences
  with a low off-peak autocorrelation have huge constraints on the length of
  the generated sequences. To overcome this limitation, exhaustive searches
  through the possible permutations have been conducted in the past. This
  exhaustive searchs have, if not properly optimized, a search space of
  $O(2^n)$, where $n$ is the length of the binary sequence, which make their
  results very limited by computational complexity.\\

  Some developments have been conducted in the past for aperiodic
  autocorrelation optimization. Even though this project searches for periodic
  autocorrelation rather than aperiodic ones, it's worth taking a look
  at these works.\\

  First of all, aperiodic autocorrelation will be defined to understand
  the optimizations proposed in these papers:\\
  \begin{definition}[Aperiodic autocorrelation]
      Given a binary sequence $S$, its aperiodic autocorrelation is defined as:
      \begin{equation}
        A'_{\tau}(S) = \sum_{i=1}^{N-\tau}s_{i}s_{i+\tau}
      \end{equation}
  \end{definition}

  All these works focus on finding a sequence $S$ that minimizes:

  \begin{definition} Given a binary sequence $S$, the energy of $S$ is defined
    as:
    \begin{equation}
      E(S) = \sum_{i=1}^{N-1} A'^{2}_{k}(S)
    \end{equation}
    \begin{equation}
      E_{min} = \operatorname*{min}_{subset} \sum_{i=1}^{N-1} A'^{2}_{k}
    \end{equation}
  \end{definition}

  One of the first optimizations for this method was
  proposed by \citet{Mertens_1996} in which he provided an algorythm
  with a complexity of $O(1.85^n)$. In his work, he applied a branch and bound
  algorythm that rules out the equivalent sequences and sets a minimum bound
  to the autocorrelation based on how complementing a single symbol of the
  sequence affects the autocorrelation.\\

  First of all, the recursion is done by picking a sequence and recursivily
  fixing elements at the extremes of the sequence as shown in Figure
  \ref{prn_search:fig:1}.\\

  \begin{figure}[ht!]
    \includegraphics[scale=0.6]{Chapters/prn_search/branching_example.png}
    \caption{An example of the branching used in \citet{Mertens_1996} with
    $N = 6$ and where x represent unfixed values.}
    \label{prn_search:fig:1}
  \end{figure}

  It's trivial that, when a symbol of the original sequence is complemented,
  the components of the autocorrelation can be lowered by, at most, -2. Based on
  that, a relaxation of $E_{min}$ can be proposed:\\
  \begin{equation}
    E_b = \sum_{k=1}^{N-1}max\{b_k, (|A'_k| - 2f_k)^2\} \leq E_{min}
  \end{equation}
  where $A'_k$ is the autocorrelation of an arbitrary sequence, $b_k = (N -
  k) \bmod 2$ the minimum possible value for $|A'_k|$ and $f_k$ the number of
  unfixed elements in $A'_k$ given by:
  \begin{equation}
    f_k = \left\{\begin{array}{ll}
        0 & k \geq N - m \\
        2(N - m - k) & N/2 \leq k < N - m\\
        N - 2m & k < N/2 \\
    \end{array}\right.
  \end{equation}

  where $N$ is the size of the sequence to search and $m$ the number of fixed
  elements at the extremes of the sequence.\\

  If $E_b$ is greater than the best candidate for $E_{min}$ so far, that branch
  can be pruned reducing the amount of computation. With this algorythm,
  Mertens optimized sucessfully up to $N = 48$.\\

  This work was further improved in \citet{Packebusch_2016}. In this paper,
  they review 2 bounds provided by different authors(Prestwich and
  Wiggenbrock) and combine them to create a new bound that lowers the
  complexity to $O(1.729^N)$, solving the LABS problem up to $N = 66$. This
  record was broken by \citet{anatoli} by computing it up to $N = 85$.\\

  \section{Our approach}

  To tackle the huge complexity encountered in the previous methods, a different
  approach was taken. Instead of dealing with the combinatorial explosion
  of all the possible binary sequences of length $N$, we decided to work with
  a smaller set consisting on all the possible sequences which can be
  constructed through the composition method with Legendre base sequences.\\

  This approach haves some pros and cons. First of all, the search space is
  reduced from $O(2^N)$ to $O(p^m)$ where $p*m = N$. This clearly means that
  the complexity grows much more smoothly than in previous works. In fact, the
  autocorrelation function can be optimized for sequences generated through the
  composition method as shown in a following chapter.\\

  The problem is that the possible sizes for the sequences are limited as $n$
  and $p$ are required to fulfill $gcd(p, m) = 1$. Even though it means that
  these method cannot find optimal sequences for all lengths, the restriction is
  looser than the non-exhaustive methods. Apart from that, this method doesn't
  explore all possible permutations and it doesn't ensure to find a pseudonoise
  sequence if it exists. To sum up, this method has been proven useful
  through examples as a good way to construct useful sequences but shouldn't
  be used to prove the non existence of pseudonoise sequences for a given
  length.\\

  Given a base sequence of size $n$ and a length $m$ for the shift sequences,
  our program needs to find all the shift sequences that generate a composite
  sequence with a good autocorrelation.\\

  This means that the search space are all the posible permutations of the
  shift sequence, in other words, $n^m$ permutations. However, there are some
  relations between the different shift sequences that let us narrow the
  search space.\\

  For example, if a constant is added to every component to the shift sequence,
  shifted version of the same sequence is obtained. This means that if only
  the permutations that start with the same component are computed,
  the whole search space would be covered as any other permutations would just
  be shifts of one permutation from the computed set. This optimization narrows
  our search space to $n^{(m-1)}$.\\

  Other optimization arises from the form of the shift sequences. In general,
  if the symbols are repeated often, they trend to generate higher
  autocorrelation spikes or periods inside the composite sequence. This
  concept can be easily expressed with the Hamming autocorrelation function:\\

  \begin{definition}[Hamming autocorrelation]
    Given a sequence $S$ of length $n$ and the function $shift$ defined at
    Equation \eqref{eq:3}, the Hamming autocorrelation is defined as:
      \begin{equation} \label{hamming:eq:1}
        HA(S)_{\tau} = \sum_{i=1}^{n} HAComponent(S_{\tau}, shift(S, \tau)_{\tau})
      \end{equation}
    where $HAComponent$ is defined as:
      \begin{equation}
        HAComponent(c1, c2) = \left\{\begin{array}{lr}
            1  &  c1 = c2\\
            0  & \textnormal{otherwise} \\
        \end{array}\right.
      \end{equation}
  \end{definition}

  For our branch and bound algorythm, it's important to note that if a symbol
  that only appears once is substituted for another, the hamming
  autocorrelation won't get lower. This means that if a depth-in-first
  bounding of the nodes that have a hamming autocorrelation higher than the
  threshold (we mean, the maximum non trivial component) is performed,
  all nodes in that branch are ensured to have a higher hamming autocorrelation
  than the threshold.\\

  \begin{figure}[ht!]
    \begin{center}
      \includegraphics[scale=0.4]{Chapters/Implementation/Example_branch_bound.png}
    \end{center}
    \caption{An example of the branch and bound algorythm with a threshold for
    hamming autocorrelation of 1 and a base sequence of length 3. Red nodes
    represent prunes and green ones final nodes in which the
    autocorrelation is computed and checked. Negative values represent those
    that haven't been initialized yet.}
    \label{bb:fig:1}
  \end{figure}

  Several things can be deduced from Figure \ref{bb:fig:1}:
  \begin{itemize}
    \item The number of autocorrelations computed can be reduced by a significant
    amount.
    \item The computation on each branch isn't balanced. This must be taken
    into account when we design the parallelism model.
  \end{itemize}


\chapter{Software Engineering}

Before starting to develop our software blindly, a good engineer has to plan
beforehand how the proccess will be. Every different project have different
characteristics that influence the methodology that should be taken.
Technologies, procedures, testing, time schedules, client meetings. All of
them must be taken into account to prevent extra work, bad quality software or
client dissatisfaction. In this chapter, most of the information is taken
directly from \citet{Sommerville}.

\section{Overall description}

  In this section we will provide a general idea behind the software project.
  \subsection{Project description}

  This project is developed to assist in the research of binary sequences with
  good autocorrelation functions. This sequences are used in application such
  as radar, wifi or GPS.\\

  The primary focus will be understanding the problem domain to create a
  software as useful as possible. Then, create an extensible program that
  implements an exhaustive search to look for new sequences to add to the
  existing scientific literature.\\

  \subsection{Product functions}

  We will summarize in a general way the functionalities we expect from
  this software.

  \begin{itemize}
    \item Perform computations in search for sequences.\\
            \begin{itemize}
              \item Obtain real-time information of the computation proccess
              \item Change dedicated resources to computation
            \end{itemize}
    \item Manage the set of found sequences
            \begin{itemize}
              \item Query by different parameters of the sequences
              \item Handle found sequences manually
              \item Plot sequence properties
            \end{itemize}
    \item Manage permissions
  \end{itemize}

  \subsection{User classes and characteristics}

  Users of this project can be divided in 2 main groups:\\

  Researchers that will control the search for new sequences, being capable of
  starting computations and modifying the database. This researchers are highly
  specialized personal and are expected to understand common abbreviations in
  the field as well as knowing which parameters are more promising to get a
  successful computation.\\

  System administrators that will optimize the application according to the
  specificiations of the system in which it is deployed. System administrators
  are specialized professional which know how to run benchmarks, maximize
  performance and understand advanced concepts of parallel computing. They are
  expected to be capable of optimizing the software with a manual explaining
  the inner workings of the program in a technical jargon.\\

  In the real world, both roles might overlapp in the same person.\\

  \subsection{Operating Enviroment}

  The computation module is expected to run in highly parallelized systems such
  as super-computers. This is the enviroment in which researchers work so it
  would be a shame if we didn't design the software to take full advantage of
  the capabilities of these systems.\\

  Taking his into account, we should target the software installed in those
  enviroments. This leads us to assume a Linux enviroment(as most supercomputers
  run it) and a Python interpreter(as it's a popular language in the scientific
  community).\\


\section{Software requirements}

In this section the software requirements of our project will be introduced.  Interviews and document analysis have been the main source to define requirements. Later they have been validated with the intended users.

\subsection{Functional requirements}

The functional requirements of the project are shown in Figure
\ref{functional:fig}.


\begin{figure}[ht!]

  \begin{center}
    \begin{tabular}{||c | p{12cm}||}
      \hline
      Identifier & Description \\
      \hline
      \hline
      FR01 \label{FR01} & The user must be capable to set the base
      sequence to use for the search \\
      \hline
      FR02 \label{FR02} & The user must be capable to set the size
      of the shift sequence to use for the search\\
      \hline
      FR03 \label{FR03} & The user must be capable of setting the maximum
      autocorrelation they are interested in \\
      \hline
       FR04 \label{FR04new} & The user must be capable of store, recover and manage sequences.\\
      \hline
      FR05 \label{FR04} & The user must be capable to extract partial
      results while the computation is still running \\
      %(this doesn't mean that the data must be avalaible with a low latency)
      \hline
      FR06 \label{FR05} & The system administrator must be capable of changing
      the resources assigned to the program\\
      \hline
      FR07 \label{FR06} & The system must provide an interface that shows
      the progress of the computation (No need for low latency
      as it would conflict with \hyperref[NFR01]{NFR01})\\
      \hline
      FR08 \label{FR07} & Software must
      run without supervision  after parameters are established. \\
      \hline
      FR09 \label{FR08} & The parameters of the load balancer must be editable
      by the system administrator in order to adapt to different hardware.
      \\
      \hline
      FR09 \label{FR09} & An system administrator must be capable of managing
      access privileges to the database and running computations\\
      \hline
      FR10 \label{FR10} & The system must provide a way to queue concurrent
      searches \\
      \hline
    \end{tabular}
  \end{center}

  \caption{Functional requierements}
  \label{functional:fig}
\end{figure}

\subsection{Non-functional requirements}

The non-functional requirements of the project are shown in Figure
\ref{non_functional:fig}.

\begin{figure}[ht!]

  \begin{center}

    \begin{tabular}{||c | c | p{7cm} | c||}

    \hline
    Identifier & Type  & Description & Relevance\\
    \hline
    \hline
    NFR01 \label{NFR01} & Performance & As sequences are being searched in a
    huge space, speed is a top priority. &
    Very high\\
    \hline
    NFR02 \label{NFR02} & Compability & As our project is being deployed in a
    environment with different architectures, it should be made as
    compatible as posible. &
    Medium \\
    \hline
    NFR03 \label{NFR03} & Usability & This software is expected to be used by
    specialized personnel. Time spent by users interacting with the application is low. Interface should be minimal. & Very low \\
    \hline
    NFR04 \label{NFR04} & Arquitecture & The program must take full advantage
    of the capabilities of supercomputers, in particular  high parallelization of the system. & High\\
    \hline
    NFR05 \label{NFR05} & Robustness & The program must handle errors such as
   data corruption, miscalculations or precision errors to avoid incorrect results. & Very high \\
    \hline
    NFR06 \label{NFR06} & Robustness & The program cannot have memory leaks that may cause a crash. Restarting memory leaked threads without affecting the
    end result & Medium \\
    \hline
    NFR07 \label{NFR07} & Extensibility & Parts of the software should be reusable to suit each research project needs. & Medium \\
    \hline
    NFR08 \label{NFR08} & Data availability & The availability of the data
    is not  a main concern as the project does not aim to be an interactive
    platform. & Low \\
    \hline
    NFR09 \label{NFR09} & Robustness & The persistence layer must be robust
    enough to avoid data loses since it is costly to produce a result. & High \\
    \hline

    \end{tabular}

  \end{center}

  \caption{Non-functional requirements}
  \label{non_functional:fig}
\end{figure}

\subsection{User interface requirements}

UI design is an important topic of software engineering as the success of a
project is related directly to the users experience and how they relate to the
software.\\

First of all, note that the users are expected to be experienced in the use of
computers, so a complex UI should not be a problem. In this case, even though
the easier the better, our development has a huge constraint on UI design that
should be taken into account: the special type of OS this project
will be deployed in. As the main focus are supercomputers, a minimalist environment
with no graphical desktop is to be expected.\\

For this reason, a command-line interface (CLI) is preferred over other alternatives as it requires fewer system resources, allows scripting and commands can be entered more rapidly as text. 
The inferface is divided in two different  main sections:\\
\begin{itemize}
  \item Application launcher (resource allocation, parallelism model, etc.)
  \item Runtime interaction with the system (tasks management, database
  queries, etc.)
\end{itemize}

As most supercomputers run UNIX-based systems, our application should follow
the POSIX\cite{POSIX_arguments} standard on the way it treats arguments.
It should follow conventions such as the use of flags such as --help or
--verbose and providing a man page.\\

It will also provide the capability to store the configuration of the system in case
the application must be restarted quickly (mainly platform specific
configuration such as parameters of the load balancer). \\


\section{Verification}
  One of the most important parts of software development is verifying the
  software. In other words, checking that the semantics of the program built
  are the same as the intended ones. Testing since the early stages of a project
  is mandatory if a quality product and an efficient development
  process is to be accomplish.  A lot of time can be invested building a
  program to realize that it does not work or a fix might generate side effects that break other
  parts of the program and so on.
  \subsection{Unit tests}

    Unit testing is the smallest piece of test suite in a project. There exist
    several approaches in the literature such as white box and black box
    testing. In our project, a mixed approach will be taken depending on the
    situation:

    \subsubsection{Property based tests}

    Property-based testing is a not so well known type of black box testing that
    is built around the idea of defining properties of functions instead of
    test cases. Originally implemented by the Haskell's library
    "QuickCheck"\footnote{
    https://hackage.haskell.org/package/QuickCheck}, this paradigm excels at
    generating huge volumes of test cases with just some extra lines of code leading
    to improvements on the coverage over the search space. It is similar to the
    test automation explained by \citet{Sommerville} in Chapter 23, being the
    main difference that an oracle that predicts the value is not needed.
    Instead, just a property of the output is checked.\\

    A well implemented library(there are several of them, but in our case we are
    working with Hypothesis\footnote{https://hypothesis.readthedocs.io/en/latest/}
    since our project is built in Python) should be capable of applying most well
    practices of black box testing, such as edge cases, all pairs, etc.\\

    The main reason why this type of test suites were chosen is that all the
    properties are already defined in this document and can be used
    straightforward as test cases. In fact, as most of our functions
    are static and pure, the generator will be very simple so the
    tests will take full advantage of this paradigm. In Figure \ref{test_example},
    there is have an example of a property used for testing the codebase.\\

    \begin{figure}[ht!]
      \inputpython{Chapters/software_engineering/test_example.py}{0}{100}
      \caption{An example test for Corollary \ref{autocorrelation:coro:1}}
      \label{test_example}
    \end{figure}

    The problem with this paradigm is that it becomes way too complicated when
    the tested methods have side effects, IO, state machines, etc. As this
    kind of systems usually depend on complex rules to build the generator of
    all the components involved in this systems. For these kind of tests, we
    will rely on the old method of designing test cases by hand.\\

\section{Validation}

The validation proccess in this project depends highly in the iteration in which
we are:

\begin{itemize}
  \item In early iterations, the validation proccess might not be as important
  as the verification proccess. This is because the core functionality of the
  program is an algorithm expressed in a technical manner with little margin
  to misinterpretations.
  \item In later iterations, the validation proccess gains weight in respect of
  the verification proccess as we dive into the UI design. In this case, there
  is more room for missunderstandings between client and developer so we must
  take this proccess into account.
\end{itemize}

Fortunatly, we are working with an agile mindset so a validation
session can be performed often so that the developers introduce the new features to
our client. Then, they can try out the features and point out missunderstandings,
desired changes, etc.\\


\section{Agile development}
    \subsection{Role definition}
    \subsection{Iterations}
      \subsubsection{Iteration 1}
      \subsubsection{Iteration 2}


\chapter{Technology choices}

  In this chapter, we will take a look at the technologies used in this project,
  the reasons behind it's adoption, pros and cons for this project and
  difficulties encountered during development.

  \section{SageMath}

SageMath\cite{Sage} is a Python mathematical suite used in research projects as
an enviroment for prototyping algorythms or math concepts in general.\\

In our case, it served as a junction point between a mathematician that is used
to express ideas in math expressions and a developer that is used to
understanding concepts by making them work. Apart from that, it was also
useful at generating some figures for this document.\\

For our use case, a way to share the notebooks through the cloud was needed.
We decided to work with a free version of CoCalc\cite{cocalc}. Even though it
served the purpose of sharing code without the need of using a repository,
I have to say that in terms of other services such as running the notebooks
was very dissapointing. For low demanding tasks it performs well, but for
bigger computations I had to copy the code and run it locally. In future
projects, I might try out the payed version (as it has tons of features)
or other alternatives.\\

  \section{CPython}
  \section{Cython}
  \section{PostgreSQL}
  \section{GIT}


\chapter{Implementation}
  In this chapter, we will introduce some specifications of the implementation
  of our solution.\\


  \section{General autocorrelation function}
    The autocorrelation function introduced in Equation \ref{eq:4} can be
    implemented in several ways:

      \subsection{Naive approach}
        The naive approach for the autocorrelation function consists in
        computing all the displacements of the sequence and then
        their correlation with the base sequence as shown in Figure
        \ref{naive_auto:fig:1}. Even though this algorythm is simple and follows
        the mathematical definition, is too slow. We have to compute the
        correlation function for each component leading us to a complexity of
        $O(n^{2})$ where n it the size of the sequence with a huge constant
        as we have to build the shifted sequence for each component.\\

        This constant can be improved if we avoid building the shifts(just
        using slices of the array) but the complexity would stay the same.

      \subsection{Wiener–Khinchin theorem}
        [Citation needed]
        The other option is to step in the world of mathematical properties.
        Fortunatly, there exists the convolution theorem that lets us express
        the autocorrelation function in terms of fourier transforms as:
        \begin{theorem}
          Given a sequence $S$, the Discrete Fourier Transform($DFT$):
          \begin{equation}
            A(S) = DFT^{-1}[DFT\{S\} · DFT\{S*\}]
          \end{equation}
          where $S*$ represents the complex conjjugate of $S$. As we are
          focusing on binary sequences, $S = S*$ so we can simplify the
          algorythm as:
          \begin{equation}
            A(S) = DFT^{-1}[DFT\{S\} · DFT\{S\}]
          \end{equation}
        \end{theorem}



          \begin{figure}
            \inputpython{Chapters/Implementation/naive.py}{0}{100}
            \caption{An example implementation of the naive autocorrelation}
            \label{naive_auto:fig:1}
          \end{figure}
      \subsection{Specific solution for the composition method}




  \section{Search space}
  \section{Parallelism model}


\chapter{Future work}

  Even though we have a working prototype that can perform the algorythm,
  there's still a lot of work to be done:\\

  First of all, the persistency part of the program needs a complete rewrite.
  Nowadays, the search results are dumped in plain text files and have no
  kind of advanced queries or a way to store sequence properties.\\

  The program has a design limitation on the diversity of base sequences
  that can be used for the computation. It would be interesting to be capable
  of using different kinds from Legendre sequences. This leads us to the topic
  of a whole refactor of the code if Domingo wants to publish it as part as
  the research project he is doing. Some parts can be written more clearly
  and the MPI usage isn't the most efficient one. For example, all the tasks
  are assigned unbuffered which means that the interconnection network delay
  hasn't been mitigated at all.\\

  Apart from that, the program haven't been tested in a production enviroment.
  Some incompatibility problems must be fixed between the development
  enviroment in which, among other things, OpenMPI was used rather than
  MPICH and I wasn't able to compile MPI4PY in the supercomputer(I have to
  say that I hadn't much time to do so because of limitations out of
  my control). Needless to say that this means we haven't extended the
  existing literature on the topic yet.\\

  To finish my contribution to Domingo's research, I need to develop a
  complexity analysis of the algorythm in a similar fashion to the previous
  works of exhaustive research. To do so, we need to run the code for a long
  time to gather enough data to estimate the said complexity.\\

  Last but not least, the UI needs some tweaking too. We should provide the
  possibility of storing the parameters in a configuration file as some of
  them will be shared between most of the tasks (for example, the number of
  processes used). I should also review if the program really needs a
  permissions system for the database. As its running in an isolated
  enviroment to perform the computation, no external user has access to the
  data unless it's uploaded to an external server through the VPN(which would
  be a terrible practice and it would be another different app).\\

  This seems like a really long lists of TODOs. Keep in mind that this
  Bachelor thesis is a part of a research project that isn't yet finished. I
  think I've accomplish the goals that my director had for this thesis.

\chapter{Conclusions}

  The final results of the project can be sumarized as:
  \begin{itemize}
    \item I have acquired enough knowledge of the problem domain
    to be capable of helping to the researchers in their software needs.
    \item I have extended my knowledge in parallel computing by learning a
    new paradigm (Message Passing Parallelism).
    \item I have learned new tools (Cython and MPI) to fulfill the
    project needs.
    \item I have adapted my knowledge in functional programming and its
    robustness to develop the tests for this software.
    \item I have applied my knowledge in algorithmic complexity to make
    rational decisions on which is the best approach to a problem.
    \item I have overcome the limitations imposed by the current health crisis
    by changing my project management methodology to an agile development.
    \item I have learned how to manage a huge volume of scientific literature
    to carry out a research.
    \item I have started to work with a supercomputer to be capable of
    deploying this software and get actual results.
  \end{itemize}

  To conclude this report, I would like to provide some final thougts:
  \begin{itemize}

    \item After dealing with Cython, I have to say that I encountered many
    problems with the compiler. The GIL checker detected GIL usages in pure C
    code. The fused types are very cryptic. If I had to do another project
    with similar characteristics, I would probably write the core modules in
    C++ (To be able to use templates) and use Cython to create the
    bindings for Python.

    \item I recognize that I should have done some test in the supercomputer
    before starting the software engineering to check the availability
    of tools and the services that the supercomputer provides by itself.

    \item Even though I said I would try to develop a portable solution, as
    far I can tell from what I've worked so far with Calderon, i think that
    wasn't a realistic objective as the programs and tools available in each
    supercomputer seem to vary a lot.

    \item I have to be more methodic in my usage of GIT. Luckly I didn't need
    to do a rollback because my commits were huge and some of them in unstable
    states.

  \end{itemize}




\backmatter
% Indique aquí el fichero .bib que contenga su bibliografía.
\bibliography{refs}

\end{document}
