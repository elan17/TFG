\section{Cython}

Python as a language comes in several implementations. CPython as the reference
implementation has it's flaws, mainly it's performance issues. As the
software being developed has performance constraints, using just the reference
implementation is not an option.\\

Fortunatly, there are alternative implementations such as Jython, IronPython,
etc. In our case, Cython\cite{Cython}\cite{Cython_book} was the chosen one as
it provides a compiler to build C code with pseudo-Python. Python code can be
called from C functions an viceversa, proving useful when a system
with C level performance to operate with high-level Python libraries is needed.\\

To support the decision of using Cython for the critical parts of the code,
some benchmarks were developed to test the actual performance improvements. As
shown in figures \ref{Cython:fig:1} and \ref{Cython:fig:2}, Cython benefits
a lot from tasks that requires iterations, but when using vector arithmetics
with Numpy the performance impact drops. This is because behind the scenes
Numpy functions are just Python wrappers for C functions so the heavy
computation is done in C.\\

\begin{figure}[ht!]
  \begin{center}
    \begin{tabular}{l r r}
      CPython:  & 318.65 seg & 100.0 \% \\
      Cython:   & 12.64 seg  & 3.9 \%   \\
      C:        & 12.33 seg  & 3.8 \%   \\
    \end{tabular}
  \end{center}
  \caption{Results of a benchmark of a long iteration.}
  \label{Cython:fig:1}
\end{figure}

\begin{figure}[ht!]
  \begin{center}
    \begin{tabular}{l r r}
      CPython:            & 8.66 seg & 100.0 \% \\
      Cython unoptimized: & 8.64 seg & 099.7 \% \\
      Cython optimized:   & 8.26 seg & 095.3 \% \\
      Cython GSL:         & 8.24 seg & 095.1 \%
    \end{tabular}
  \end{center}
  \caption{Results of a benchmark of vector operations using Numpy or GSL.}
  \label{Cython:fig:2}
\end{figure}

One might think that, if vector operations are so efficient in CPython, it would
be simpler to just use Numpy methods to implement our algorythms (which was
indeed done in the general autocorrelation function with the algortyhm
based on the convolution theorem). The problem raises when it is needed to
access the Numpy array in an undefined way by the library and to code a loop to
compute a function (the composite autocorrelation for example).\\

Cython provides native support for Numpy arrays, letting us access them with a
C level performance. In fact, as it supports fused types (the equivalent to
templates in C++), functions that can work with diverse types
depending on the input arguments can be defined. Even though in our case it will
be skipped as it comes with extra headaches and only integers are needed for the
purposes of this project, it's a nice feature if it was needed to extend Sage to
fully support our research field.\\

One important thing to take into account when developing with Cython is that
C types allocated in the heap (Cython supports raw C vectors and data structures
from C++ std) doesn't have automatic memory management. This will make the
debugging tougher as it will be needed to look for memory leaks. In contrast,
Numpy arrays do support automatic memory management at the cost of the overhead
of type checking and reference counting.\\
