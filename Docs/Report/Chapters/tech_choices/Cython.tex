\section{Cython}

Python as a language comes in several implementations. CPython as the reference
implementation has it's flaws, mainly because of it's performance issues. As we
are developing a software with performance constraints, using just the reference
implementation is not an option.\\

Fortunatly, there are alternative implementations such as Jython, IronPython,
etc. In our case, we are going to use Cython\cite{Cython}\cite{Cython_book} as
it provides a compiler to build C code with pseudo-Python. We can even call
python code from C functions an viceversa, proving useful when we need a system
with C level performance to operate with high-level Python libraries.\\

To support the decision of using Cython for the critical parts of the code,
we designed some benchmarks to test the actual performance improvements. As
shown in figures \ref{Cython:fig:1} and \ref{Cython:fig:2}, Cython benefits
a lot from tasks that requires iterations, but when using vector arithmetics
with Numpy the performance impact drops. This is because behind the scenes
Numpy functions are just Python wrappers for C functions so the heavy
computation is done in C.\\

\begin{figure}[ht!]
  \begin{center}
    \begin{tabular}{l r r}
      CPython:  & 318.65 seg & 100.0 \% \\
      Cython:   & 12.64 seg  & 3.9 \%   \\
      C:        & 12.33 seg  & 3.8 \%   \\
    \end{tabular}
  \end{center}
  \caption{Results of a benchmark of a long iteration.}
  \label{Cython:fig:1}
\end{figure}

\begin{figure}[ht!]
  \begin{center}
    \begin{tabular}{l r r}
      CPython:            & 8.66 seg & 100.0 \% \\
      Cython unoptimized: & 8.64 seg & 099.7 \% \\
      Cython optimized:   & 8.26 seg & 095.3 \% \\
      Cython GSL:         & 8.24 seg & 095.1 \%
    \end{tabular}
  \end{center}
  \caption{Results of a benchmark of vector operations using Numpy or GSL.}
  \label{Cython:fig:2}
\end{figure}

One might think that, if vector operations are so efficient in CPython, we could
just use Numpy methods to implement our algorythms (which we indeed did in
the general autocorrelation function with Wiener–Khinchin's algorythm). The
problem raises when we need to access the Numpy array in an undefined way and
we have to code a loop to compute a function (the composite autocorrelation
for example).\\

Cython provides native support for Numpy arrays, letting us access them with a
C level performance. In fact, as it supports fused types (the equivalent to
templates in C++), we can define functions that can work with diverse types
depending on the input arguments. Even though in our case we will skip it
as it comes with extra headaches and we are working with just integers,
it's a nice feature if we would like to extend Sage to fully support our
research field.\\

One important thing to take into account when developing with Cython is that
C types allocated in the heap (Cython supports raw C vectors and data structures
from C++ std) doesn't have automatic memory management. This will make the
debugging tougher as we will need to look for memory leaks. In contrast,
Numpy arrays do support automatic memory management at the cost of the overhead
of type checking and reference counting.\\
