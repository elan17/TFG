\appendix
\section{An overview on finite fields}
The particular type of LFSR used in m-sequences can be simulated with extensions of binary
Finite Fields.  A formal definition is that m-sequences are a trace mapping from an extension field $GF(p^d)$ to a base field $GF(p)$. The base field consists of the integers modulo $p$ with $p$ prime. In this work, we will restrict ourselves to binary sequences $p=2$, that have attracted the most interest.  As shown in \citet{golomb_ref}, a sequence $a(i)$, where $i$ is an indexing integer, is formed by taking the elements of the extension field in consecutive increasing powers of $\alpha$, a primitive element, and computing  the trace to the base field.

\begin{definition}[Finite Fields]
  Given $E/GF(2)$, $\alpha$ a primitive element of $E$ and $S$ the resulting
  sequence:
  \begin{equation}
    S_{i} = Tr(\alpha^{i})
  \end{equation}
\end{definition}
 The base field consists of the integers modulo $p$ with $p$ prime. If $\beta$ is an element of $GF(p^d)$ then
\begin{equation}
 Tr^d(\beta)=\sum_{i=0}^{d-1} \beta^{p^i}.
\end{equation}

\begin{figure}[ht!]
  \inputpython{Chapters/PRN_generation/example_mls.py}{0}{100}
  \caption{An example of a posible implementation of m-sequences}
  \label{mls:fig:1}
\end{figure}
 
