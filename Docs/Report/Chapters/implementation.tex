\chapter{Implementation}
  In this chapter, we will introduce some specifications of the implementation
  of our solution.\\


  \section{General autocorrelation function}
    The autocorrelation function introduced in Equation \ref{eq:4} can be
    implemented in several ways:

      \subsection{Naive approach}
        The naive approach for the autocorrelation function consists in
        computing all the displacements of the sequence and then
        their correlation with the base sequence as shown in Figure
        \ref{naive_auto:fig:1}. Even though this algorythm is simple and follows
        the mathematical definition, is too slow. We have to compute the
        correlation function for each component leading us to a complexity of
        $O(n^{2})$ where n it the size of the sequence with a huge constant
        as we have to build the shifted sequence for each component.\\

        This constant can be improved if we avoid building the shifts(just
        using slices of the array) but the complexity would stay the same.

      \subsection{Wiener–Khinchin theorem}
        [Citation needed]
        The other option is to step in the world of mathematical properties.
        Fortunatly, there exists the convolution theorem that lets us express
        the autocorrelation function in terms of fourier transforms as:
        \begin{theorem}
          Given a sequence $S$, the Discrete Fourier Transform($DFT$):
          \begin{equation}
            A(S) = DFT^{-1}[DFT\{S\} · DFT\{S*\}]
          \end{equation}
          where $S*$ represents the complex conjjugate of $S$. As we are
          focusing on binary sequences, $S = S*$ so we can simplify the
          algorythm as:
          \begin{equation}
            A(S) = DFT^{-1}[DFT\{S\} · DFT\{S\}]
          \end{equation}
        \end{theorem}



          \begin{figure}
            \inputpython{Chapters/Implementation/naive.py}{0}{100}
            \caption{An example implementation of the naive autocorrelation}
            \label{naive_auto:fig:1}
          \end{figure}
      \subsection{Specific solution for the composition method}




  \section{Search space}
  \section{Parallelism model}
