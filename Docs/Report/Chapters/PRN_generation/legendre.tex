\section{Legendre sequences}

Legendre sequences, as explained in \citet{legendre_sequences}, are binary
sequences defined through quadratic residues as follows:

\begin{definition}
  Let $p$ be an odd prime and the function "Legendre Symbol" be:
    \begin{equation}
      LSy(n, p) = \left\{\begin{array}{lr}
          1  & \textnormal{if n is a quadratic residue mod p}   \\
          -1 & \textnormal{otherwise} \\
      \end{array}\right.
    \end{equation}
  We define the Legendre Sequence as:
    \begin{equation}
      LSs(p)_{i} = LSy(i, p) \textnormal{  where  } 0 \leq i < p
    \end{equation}
\end{definition}


An example implementation could be:
\inputpython{Chapters/PRN_generation/example_legendre.py}{0}{100}

Some Legendre Sequences have interesting autocorrelation properties:
\begin{property}\label{property:2.3.1}
  Given an odd prime $p$ such as $p \equiv 3 \bmod 4$, we can say that
  $LSs(p)$ has a flat autocorrelation.\cite{legendre_sequences}
\end{property}

Even though \citet{legendre_sequences} has a generalization of property
\ref{property:2.3.1} to all Legendre Sequences, it requires the introduction of
a third symbol making the sequence non-binary. \\

As $p$ is the variable defining the size of the generated sequence, we can
derive that the distribution of Legendre Sequences is related to the Prime
Number Theorem. This means that Legendre Sequences have more possible lengths
than MLS or other exponential contructions. However, Legendre Sequences have
the drawback that there is only on per sequence length and they cannot generate
families of good autocorrelation properties by themselfs.
