\section{Legendre sequences}

Legendre sequences, as explained in \citet{legendre_sequences}, are binary
sequences defined through quadratic residues. A quadratic residue modulo $p$ is an integer number $x$ between $0$ and $p-1$ such as exist an integer y satisfying $x=y^2 \bmod p$.

\begin{definition}
  Let $p$ be an odd prime and the  Legendre Symbol function  be
    \begin{equation}
      LSy(n, p) = \left\{\begin{array}{lr}
          1  & \textnormal{if n is a quadratic residue mod p}   \\
          -1 & \textnormal{otherwise} \\
      \end{array}\right.
    \end{equation}
  The Legendre sequence can be defined as:
    \begin{equation}
      LSs(p)_{i} = LSy(i-1, p) \textnormal{  where  } 0 \leq i \leq p-1
    \end{equation}
\end{definition}

\begin{figure}[ht!]
  \inputpython{Chapters/PRN_generation/example_legendre.py}{0}{100}
  \caption{An example implementation of the generation of a Legendre sequence.}
  \label{}
\end{figure}
 Legendre sequences with length  satisfying $p=3\bmod 4$ have flat autocorrelation, i.e. 
 the value of the autocorrelation for any nonzero shift is $-1$. There is a generalization of this property by \citet{legendre_sequences}  but it requires the sequence 
 to be non-binary.\\ 

As $p$ is the length of the generated sequence, the distribution of Legendre sequences is related to the Prime Number Theorem. This means that Legendre sequences have more possible lengths than in m-sequences or other constructions based on them. However, Legendre sequences have the drawback that there exists only one per sequence length.
