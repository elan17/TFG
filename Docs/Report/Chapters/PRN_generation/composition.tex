\section{Composition method}

\subsection{Algorythm}

The composition method (introduced by \citet{tirkel_composition} as Prime
Arrays) uses a base sequence and a sequence of shifts to create a matrix of
sequence components as follows:

\begin{definition}[Composite matrix]
  Given a base sequence $S$ of length $n$ and a sequence of integers $T$ of
  length $m$ such that:
  \begin{equation}\label{composition:eq:1}
    0 \leq T_{i} < n
  \end{equation}
  \begin{equation}\label{composition:eq:2}
    gcd(n, m) = 1
  \end{equation}

  Given the $shift$ function defined in Equation \ref{eq:3},
  we define the composite matrix as:

  \begin{equation}\label{composition:eq:3}
    CM(S, T) = \begin{bmatrix}
      shift(S, T_{1})_{1} & shift(S, T_{2})_{1} & \dots & shift(S, T_{m})_{1} \\
      shift(S, T_{1})_{2} & shift(S, T_{2})_{2} \\
      \vdots & & \ddots \\
      shift(S, T_{1})_{n} & & & shift(S, T_{m})_{n}
    \end{bmatrix}
  \end{equation}

  In other words, each column represents a shift of the base sequence defined
  by the sequence of shifts.
\end{definition}

\begin{definition}[Composite sequence]
  Given a base sequence $S$ of length $n$ and a sequence of integers $T$ of
  length $m$ that fulfill Equations \ref{composition:eq:1} and
  \ref{composition:eq:2} and the composite matrix defined at Equation
  \ref{composition:eq:3}, we define the composite sequence as:
  \begin{equation}
    CS(S, T)_{i} = CM(S, T)_{(i \bmod m), (i \bmod n)}
  \end{equation}
\end{definition}

\begin{figure}[ht!]
  $$S = \begin{bmatrix}
    0 & 1 & 2 & 3 & 4\\
  \end{bmatrix}$$
  $$T = \begin{bmatrix}
    0 & 2 & 1 & 4 & 3 \\
  \end{bmatrix}
  $$
  $$CM(S, T) = \begin{bmatrix}
  0 & 3 & 4 & 1 & 2 & 4\\
  1 & 4 & 0 & 2 & 3 & 0\\
  2 & 0 & 1 & 3 & 4 & 1\\
  3 & 1 & 2 & 4 & 0 & 2\\
  4 & 2 & 3 & 0 & 1 & 3\\
  \end{bmatrix}
  $$
  $$CS(S, T) = \text{[0 4 1 4 1 4 1 0 2 0 2 0 2 1 3 1 3 1 3 2 4 2 4 2 4 3 0 3 0 3]}\\
  $$
  \caption{Example of a computation of the composition method(note that we are
  using a non-binary sequence to illustrate better the method).}
  \label{}
\end{figure}

We can define the correlation function for composite matrixes:

\begin{definition}
  Given 2 composite matrixes $M0$ and $M1$ with $n$ rows and $m$ columns,
  we define it's correlation as:
  \begin{equation}
    C(M0, M1) = \sum_{i=1}^{m}\sum_{j=1}^{n}M0_{j, i}M1_{j,i}
  \end{equation}
\end{definition}

We can also define the shift function for composite matrixes:

\begin{definition}
  Given a composite matrix $M$ of $n$ rows and $m$ columns, we define the
  shift function as:
  \begin{equation}
    shift(M, \tau)_{i, j} = M_{(i-\tau \bmod m), (j-\tau \bmod n)}
  \end{equation}
\end{definition}
This let's us establish the following relation:
  \begin{corollary} Getting a shift of the composite sequence is equivalent to
  applying a shift to the matrix and then extracting the corresponding
  sequence.
    \begin{equation}
      shift(CS(S, T), \tau)_{i} = shift(CM(S, T), \tau)_{(i \bmod m), (i \bmod n)}
    \end{equation}
  \end{corollary}


When we define the autocorrelation function for a composite matrix, an
interesting property arises:

\begin{definition}
  Given a composite matrix $M$ of $n$ rows and $m$ columns, we define it's
  autocorrelation function as:
  \begin{equation}
    \begin{split}
      A(M)_{\tau} = C(M, shift(M, \tau)) = \sum_{i=1}^{m}\sum_{j=1}^{n}M_{j, i}M_{(j-\tau \bmod m),(i - \tau \bmod n)} = \\
      \sum_{i=1}^{m}\sum_{j=1}^{n}shift(S, T_{i})_{j}shift(S, T_{(i - \tau \bmod n)})_{(j-\tau \bmod m)}
    \end{split}
  \end{equation}
\end{definition}

Notice that if we keep $i$ constant, we get a particular component of the
autocorrelation function of $S$.

\begin{property} \label{composition:prop:1}
We can define the autocorrelation function of the composite
sequence in terms of the autocorrelation function as follows:

\begin{equation}
  \sum_{j=1}^{n}shift(S, T_{i})_{j}shift(S, T_{(i - \tau \bmod n)})_{(j-\tau \bmod m)} = A(S)_{|((T_{j} - \tau) \bmod m) - T_{(j + \tau) \bmod n}|}
\end{equation}
\begin{equation}
  A(M)_{\tau} = \sum_{i=1}^{m} A(S)_{|((T_{j} - \tau) \bmod m) - T_{(j + \tau) \bmod n}|}
\end{equation}

\end{property}

This property will prove useful in our software project as it's a fast method
of computing the autocorrelation function.

\subsection{Costas arrays}

Costas arrays, discovered independently by John P. Costas\cite{costas_costas}
and E.N. Gilbert \cite{gilbert_costas} in 1965, are a set of sequences highly
used in radar and sonar applications. We are going to provide the 2 definitions
as both will be useful for different purposes:

\begin{definition}[Costas array(Costas)]
  Square matrix of size $n×n$ filled with 0s and 1s such that there
  aren't more than multiple 1s in each row or column and that every
  displacement vector is distinct from the rest.
\end{definition}

This definition is used in several deployments of sonar and radar to generate
systems with a good ambiguity function, in other words, tolerant to the Doppler
effect.

\begin{figure}[ht!]
  $$
  \begin{bmatrix}
   1&0&0&0\\
   0&0&0&1\\
   0&1&0&0\\
   0&0&1&0
  \end{bmatrix}
  $$
  \caption{An example of a Costas array}
  \label{fig:costas_1}
\end{figure}

Notice that we could compact the representation by just having a list of the
rows in which each 1 lives:

\begin{figure}[ht!]
  $$[4, 2, 1, 3]$$
  \caption{The compact representation of the Costas array of Figure
  \ref{fig:costas_1}.}
  \label{fig:costas_2}
\end{figure}

This representation is equivalent to the definition of a Costas array provided
by Gilbert:

\begin{definition}[Distinct difference permutations]\label{def:costas_1}
  Given a sequence of integers $S$ of length $n$ such that:
    \begin{equation}\label{eq:costas_1}
      0 \leq S_{i} < n
    \end{equation}
  We say that $S$ is a distinct difference permutation $r$ apart if,
  for any given pair $(S_{i}, S_{j})$, satisfies:
    \begin{equation}\label{eq:costas_2}
      S_{i} - S_{i+r} \not \equiv S_{j} - S_{j+r} \bmod n
    \end{equation}
\end{definition}

\begin{definition}[Costas array(Gilbert)]
  Given a sequence $S$ satisfying Equation \ref{eq:costas_1}, we say it's a
  Costas array if, for any given $r$ value, it satisfies Equation
  \ref{eq:costas_2}.
\end{definition}

This compact representation can be feeded into the composition method as
a sequence of shifts generating interesting new
sequences\cite{moreno_costas}.\\

Several construction methods have
been proposed. For sake of simplicity, we are going to introduce just the Welsh
construction as defined in Gilbert \cite{gilbert_costas}:\\

Given a prime number $p$ and a primitive root $g$ of $p$, we can contruct a
Costas array $S$ as follows:
\begin{equation}
  S_{i} \equiv g^{i} \bmod p
\end{equation}

Notice that this contruction can generate sequences of a prime length as
the Legendre Sequences. However, it can generate several sequences for a given
length. As the number of possible sequences depend on the number of primitive
elements of the finite field of order $p$, the number of possible costas arrays
for a given length using this construction is $\phi(p-1)$ where $\phi$ is the
Euler's totient function.
