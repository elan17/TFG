\section{Composition method}

\subsection{Algorithm}

The composition method  was introduced by \citet{tirkel_composition}, initially called prime
arrays,  uses a base sequence and a sequence of shifts to create a matrix  whose columns are
shifts of the base sequence as follows:

\begin{definition}[Composite matrix]
  Given a base sequence $S$ of length $n$ and a sequence of integers $T$ of
  length $m$ such that:
  \begin{equation}\label{composition:eq:1}
    0 \leq T_{i} < n
  \end{equation}
  \begin{equation}\label{composition:eq:2}
    gcd(n, m) = 1
  \end{equation}

  Given the $shift$ function defined in Equation \eqref{eq:3},
   the composite matrix is defined as

  \begin{equation}\label{composition:eq:3}
    CM(S, T) = \begin{bmatrix}
      shift(S, T_{0})_{0} & shift(S, T_{1})_{0} & \dots & shift(S, T_{m-1})_{0} \\
      shift(S, T_{0})_{1} & shift(S, T_{1})_{1} \\
      \vdots & & \ddots \\
      shift(S, T_{0})_{n-1} & & & shift(S, T_{m-1})_{n-1}
    \end{bmatrix}
  \end{equation}

%  In other words, each column represents a shift of the base sequence defined
 % by the sequence of shifts.
\end{definition}

\begin{definition}[Composite sequence]
  Given a base sequence $S$ of length $n$, a shift sequence $T$ of
  length $m$ that satisfy Equations \eqref{composition:eq:1} and
  \eqref{composition:eq:2} and the composite matrix defined at Equation
  \eqref{composition:eq:3}, the composite sequence is defined as:
  \begin{equation}
    CS(S, T)_{i} = CM(S, T)_{(i \bmod m), (i \bmod n)}
  \end{equation}
\end{definition}
The fact that the sequence has length $nm$ and the correlation function can be deduce from the matrix is a consequence of the Chinese Remainder Theorem.  This is informally called the diagonal method, because it unfolds the array following the diagonal and using the wrap around provided by the modulo operation.
\begin{figure}[ht!]
  $$S = \begin{bmatrix}
    0 & 1 & 2 & 3 & 4\\
  \end{bmatrix}$$
  $$T = \begin{bmatrix}
    0 & 2 & 1 & 4 & 3 \\
  \end{bmatrix}
  $$
  $$CM(S, T) = \begin{bmatrix}
  0 & 3 & 4 & 1 & 2 & 4\\
  1 & 4 & 0 & 2 & 3 & 0\\
  2 & 0 & 1 & 3 & 4 & 1\\
  3 & 1 & 2 & 4 & 0 & 2\\
  4 & 2 & 3 & 0 & 1 & 3\\
  \end{bmatrix}
  $$
  $$CS(S, T) = \text{[0 4 1 4 1 4 1 0 2 0 2 0 2 1 3 1 3 1 3 2 4 2 4 2 4 3 0 3 0 3]}\\
  $$
  \caption{Example of a computation of the composition method. For clarity purposes, the example uses a non binary base sequence.}
  \label{}
\end{figure}

\begin{definition}[Composite matrix correlation]
  Given two composite matrices $M$ and $R$ with $n$ rows and $m$ columns,
  its correlation is defined as
  \begin{equation}
    C(M, R) = \sum_{i=0}^{m-1}\sum_{j=0}^{n-1}M_{j, i}R_{j,i}.
  \end{equation}
\end{definition}

\begin{definition}[Composite matrix shift]\label{definition:composition:shift}
  Given a composite matrix $M$ of $n$ rows and $m$ columns, the shift function
  is defined as:
  \begin{equation}
    shift(M, \tau)_{i, j} = M_{(i-\tau \bmod m), (j-\tau \bmod n)}
  \end{equation}
\end{definition}
This means that the following relation can be established:
  \begin{corollary} Getting a shift of the composite sequence is equivalent to
  applying a shift to the matrix and then extracting the corresponding
  sequence.
    \begin{equation}
      shift(CS(S, T), \tau)_{i} = shift(CM(S, T), \tau)_{(i \bmod m), (i \bmod n)}
    \end{equation}
  \end{corollary}

The previous corollary has an interesting implication for computing the autocorrelation function.

\begin{corollary}
  Given a composite matrix $M$ of $n$ rows and $m$ columns, its
  autocorrelation function is defined as:
  \begin{equation}
    \begin{split}
      A(M)_{\tau} = C(M, shift(M, \tau)) = \sum_{i=0}^{m-1}\sum_{j=0}^{n-1}M_{j, i}M_{(j-\tau \bmod m),(i - \tau \bmod n)} = \\
      \sum_{i=0}^{m-1}\sum_{j=0}^{n-1}shift(S, T_{i})_{j}shift(S, T_{(i - \tau \bmod n)})_{(j-\tau \bmod m)}.
    \end{split}
  \end{equation}
\end{corollary}

Notice that if $i$ is kept constant, a particular component of the
autocorrelation function of $S$ is obtained.

\begin{property} \label{composition:prop:1}
The autocorrelation function of the composite
sequence can be defined in terms of the autocorrelation function as follows:

\begin{equation}
  \sum_{j=0}^{n-1}shift(S, T_{i})_{j}shift(S, T_{(i - \tau \bmod n)})_{(j-\tau \bmod m)} = A(S)_{|((T_{j} - \tau) \bmod m) - T_{(j + \tau) \bmod n}|}
\end{equation}
\begin{equation}
  A(M)_{\tau} = \sum_{i=0}^{m-1} A(S)_{|((T_{j} - \tau) \bmod m) - T_{(j + \tau) \bmod n}|}
\end{equation}

\end{property}

This property is useful  as a fast method for computing the autocorrelation function for composed sequences and this will be implemented for efficiency.

\subsection{Costas arrays}

Costas arrays, discovered independently by John P. Costas\cite{costas_costas}
and E.N. Gilbert \cite{gilbert_costas} in 1965, are a family of sequences highly
used in radar and sonar applications. The equivalent definitions from both works has been included as both will be useful for different purposes.

\begin{definition}[Costas array by Costas \cite{costas_costas})]
 An Square matrix of size $n×n$  is filled with 0s and 1s such that there is a single
 1 in each row and column. Thus if the
  displacement vector is calculated by substracting all pairs of positions of the 1s, then  the results are unique.
\end{definition}

This definition is used in several deployments of sonar and radar to generate
systems with a good ambiguity function, in other words, tolerant to the Doppler
effect.

\begin{figure}[ht!]
  $$
  \begin{bmatrix}
   1&0&0&0\\
   0&0&0&1\\
   0&1&0&0\\
   0&0&1&0
  \end{bmatrix}
  $$
  \caption{An example of a Costas array}
  \label{fig:costas_1}
\end{figure}

Notice that the representation can be compacted by just having a list of the
rows in which each 1 lives:

\begin{figure}[ht!]
  $$[3, 1, 0, 2]$$
  \caption{The compact representation of the Costas array of Figure
  \ref{fig:costas_1}.}
  \label{fig:costas_2}
\end{figure}

This representation is equivalent to the definition of a Costas array provided
by Gilbert:

\begin{definition}[Distinct difference permutations]\label{def:costas_1}
  Given a sequence of integers $S$ of length $n$ such that:
    \begin{equation}\label{eq:costas_1}
      0 \leq S_{i} < n
    \end{equation}
  It is said that $S$ is a distinct difference permutation $r$ apart if,
  for any given pair $(S_{i}, S_{j})$, satisfies:
    \begin{equation}\label{eq:costas_2}
      S_{i} - S_{i+r} \not \equiv S_{j} - S_{j+r} \bmod n
    \end{equation}
\end{definition}

\begin{definition}[Costas array by Gilbert \cite{gilbert_costas}]
  Given a sequence $S$ satisfying Equation \eqref{eq:costas_1}, it is called a
  Costas array if, for any given $r$ value, it satisfies Equation
  \eqref{eq:costas_2}.
\end{definition}

This compact representation can be feeded into the composition method as
a sequence of shifts generating interesting new
sequences\cite{moreno_costas}.\\

Several construction methods for Costas arrays have
been proposed. For sake of simplicity, just the Welch
construction will be introduced following the work of  Gilbert\cite{gilbert_costas}.

Given a prime number $p$ and a primitive root $g$ of $p$ such as all the powers of $g$ in $[0, p-1]$ modulo $p$ are different from each other.  A Costas array $S$
can be constructed as follows:
\begin{equation}
  S_{i} \equiv g^{i} \bmod p, \quad 0 \leq i <p-1
\end{equation}

This  construction can generate sequences of  length $p-1$ for a given $g$. As the number of possible sequences depend on the number of primitive
elements of the finite field of order $p$, the number of possible Costas arrays
for a given length using this construction is $\phi(p-1)$ where $\phi$ is the
Euler's totient function(see
\citet{manfred_totient}).
