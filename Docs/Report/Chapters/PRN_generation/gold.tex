\section{Gold Codes}

Gold codes\cite{gold_codes} are a family of sequences, derived from
m-sequences, with very important properties that are used in several
applications such as wireless communication and geolocalisation. A Gold Code
generator gets two m-sequences sequences that fulfill:

\begin{property}
  Given two m-sequences that can generate Gold Codes, $S1$ and $S2$, of length
  $2^{n}-1$, $CC$ the crosscorrelation function defined in Equation \ref{eq:7}:
    \begin{equation}\label{gold:eq:1}
      max |CC(S1, S2)| \leq 2^{\frac{n+2}{2}}
    \end{equation}
\end{property}

And XORs all their relative shifts generating a family of $2^{n} + 1$ sequences
($2^{n} - 1$ XORed sequences + 2 m-sequences).

\begin{property}
  Given any two sequences, S1 and S2, from a Gold family of sequences of length
  $2^{n}-1$, $CC$ the crosscorrelation function defined in Equation \ref{eq:7}:
  \begin{equation}
        max |CC(S1, S2)| \leq \left\{\begin{array}{lr}
            2^{\frac{n+2}{2}}+1 & \textnormal{ if n is even } \\
            2^{\frac{n+1}{2}}+1 & \textnormal{ if n is odd } \\
        \end{array}\right.
  \end{equation}
\end{property}

% TODO: Add a  code example

This property means that the crosscorrelation between any given pair of
sequences from a Gold family is low enough to differentiate them. This proves
useful when several devices are transmiting in the same frecuency and we must
treat signals that are not the one we want to receive as noise.\\

Gold also showed  a way to generate this pair of sequences,
using a decimation of one m-sequence.

\begin{definition}[Decimation]
  Given a sequence $S$ of length $n$, a decimation by $q$ of $S$ is defined as:
  \begin{equation}
    S[q]_{i} = S_{((q·i) \bmod n)}
  \end{equation}
\end{definition}

\begin{property}
  Given a m-sequence $S$ of length $2^n - 1$ where n is odd and and a
  coprime of $n$ named $k$, the sequence pair $(S, S[2^{k} + 1])$ number
  fulfills equation \ref{gold:eq:1}.
\end{property}

\begin{figure}[ht!]
  \inputpython{Chapters/PRN_generation/example_gold.py}{0}{100}
  \caption{An example implementation of a generation of a family of gold
  sequences relying in the example at Figure \ref{mls:fig:1}.}
  \label{}
\end{figure}

Notice that this construction has the same problem as m-sequences. It's
an exponential contruction so it might not be enough for some applications.
