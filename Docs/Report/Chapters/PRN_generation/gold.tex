\section{Gold Codes}

Gold codes\cite{gold_codes} are a family of sequences derived from
m-sequences, with good properties to several
applications such as wireless communication and geolocalization.\\

A Gold Code generator gets two m-sequences $S1$ and $S2$  of length
 $2^{n}-1$ ($n$ is a natural number), that fulfill that the crosscorrelation function $CC$ (See Equation \eqref{eq:7}) is 
  \begin{equation}\label{gold:eq:1}
      max |CC(S1, S2)| \leq 2^{\frac{n+2}{2}}
    \end{equation}

The family called Gold codes is defined by the sequence 
$$
\{S_1, S_2, S_1\oplus S_2, S_1\oplus shift(S_2, 1), \ldots, S_1\oplus shift(S_2,2^n-2)\}.
$$


\begin{property}
  Given any two sequences, $S1$ and $S2$, from a Gold family of sequences of length
  $2^{n}-1$,  the crosscorrelation function $CC$ defined in Equation \eqref{eq:7} satisfies 
  \begin{equation}
        max |CC(S1, S2)| \leq \left\{\begin{array}{lr}
            2^{\frac{n+2}{2}}+1 & \textnormal{ if n is even } \\
            2^{\frac{n+1}{2}}+1 & \textnormal{ if n is odd } \\
        \end{array}\right.
  \end{equation}
\end{property}

This property means that the crosscorrelation between any given pair of
sequences from a Gold family is low enough to differentiate them.
 In practical, several devices can transmit over the same medium, then the receiver can recover the information only if it knows the corresponding Gold code. In other case it  resembles noise.

Gold also proposed  a way to find $S1$ and $S2$ by decimation..

\begin{definition}[Decimation]
  Given a sequence $S$ of length $n$, a decimation by $q$ of $S$ is defined as
  \begin{equation}
    S[q]_{i} = S_{((q·i) \bmod n)}.
  \end{equation}
\end{definition}

\begin{property}
  Given a m-sequence $S$ of length $2^n - 1$, where n is odd and
  coprime with an integer $k$, the sequence pair $(S, S[2^{k} + 1])$ 
  satisfies Equation \eqref{gold:eq:1}.
\end{property}

\begin{figure}[ht!]
  \inputpython{Chapters/PRN_generation/example_gold.py}{0}{100}
  \caption{An  implementation of a Gold family generator family in Python}
  \label{}
\end{figure}


Notice that this construction has the same problem as m-sequences, which is that only few choices of  sequence length are possible . It may not be suitable for some applications.
