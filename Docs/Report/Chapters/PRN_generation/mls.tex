\section{Maximum Length Sequence(m-sequences)}

M-sequences are an exponential binary pseudonoise construction that was
initially concived using linear feedback shift registers(LFSR). The particular
type of LFSR used in m-sequences can be simulated with extensions of binary
Finite Fields. The definitions are:

\begin{definition}[LFSR]
  A m-sequence is a binary sequence generated by an LFSR that, given an initial
  state different from 0, it cycles between all posible states except 0.
\end{definition}

Which, as shown in \citet{golomb_ref}, is equivalent to:

\begin{definition}[Finite Fields]
  Given $E/GF(2)$, $\alpha$ a primitive element of $E$ and $S$ the resulting
  sequence:
  \begin{equation}
    S_{i} = trace(\alpha^{i})
  \end{equation}
\end{definition}

\begin{property}
  A m-sequence will always be of length of the form $2^{n}-1$ where n is an
  arbitrary natural number.
\end{property}

\begin{property}
  A m-sequence sequence will always have an autocorrelation function such as
  all the components will be -1 except when $\tau = 0$
\end{property}

\begin{figure}[ht!]
  \inputpython{Chapters/PRN_generation/example_mls.py}{0}{100}
  \caption{An example of a posible implementation of m-sequences}
  \label{mls:fig:1}
\end{figure}

Notice that, even though the construction is exponential, the complexity of
the algorithm is $O(n)$ when $n$ is the size of the sequence. The problem is
that sequences of arbitrary size might be needed in some applications. As it will
be discussed see in a following chapter, the complexity of computing the
autocorrelation with the Fourier Transform approach is $O(n*log(n))$ so using longer
sequences than needed has a direct impact on the performance of the system. \\

This sequences by themselves might not a be huge deal because they don't define
a way to build families of well crosscorrelated sequences, but they are the
building blocks for other contructions, such as the Gold Codes used in GPS and
CDMA.
