\section{Maximum Length Sequence(m-sequences)}

M-sequences are a binary pseudonoise construction that was initially conceived using linear feedback shift registers (LFSR). A m-sequence is a binary sequence generated by an LFSR that, given an initial
state different from 0, it cycles between all posible states except 0. (See Appendix for formal details)
\domingo{Aquí podrías poner una imagen de un LFSR si quieres}
The following properties hold for a m-sequences  
\begin{itemize}
\item   A m-sequence will always be of length of the form $2^{m}-1$ where $m$ is an
  arbitrary natural number.
\item A m-sequence sequence will always have an autocorrelation function such as
  all the components will be -1 except when $\tau = 0$
\end{itemize}
The complexity of the algorithm to construct an m-sequence of length $n$ is $O(n)$. The problem is
that sequences of arbitrary length might be needed in some applications.

 As it will be discussed in a following chapter, the complexity of computing the
autocorrelation with the Fourier Transform approach is $O(n*log(n))$ so using longer length than the strictly needed  has a  negative direct practical consequences.\\

These sequences by themselves might not a be huge deal because they do not define a way to build families of sequences with low cross correlation, but they are the
building blocks for other constructions, such as the Gold Codes that are used in GPS and  CDMA.
