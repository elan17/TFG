\section{Costas Arrays}

Costas arrays, discovered independently by John P. Costas\cite{costas_costas}
and E.N. Gilbert \cite{gilbert_costas} in 1965, are a set of sequences highly
used in radar and sonar applications. We are going to provide the 2 definitions
as both will be useful for different purposes:

\begin{definition}[Costas array(Costas)]
  Square matrix of size $n×n$ filled with 0s and 1s such that there
  aren't more than multiple 1s in each row or column and that every
  displacement vector is distinct from the rest.
\end{definition}

This definition is used in several deployments of sonar and radar to generate
systems with a good ambiguity function, in other words, tolerant to Doppler
effect.

\begin{figure}[ht!]
  $$
  \begin{bmatrix}
   1&0&0&0\\
   0&0&0&1\\
   0&1&0&0\\
   0&0&1&0
  \end{bmatrix}
  $$
  \caption{An example of a Costas array}
  \label{fig:costas_1}
\end{figure}

Notice that we could compact the representation by just having a list of the
rows in which each 1 lives:

\begin{figure}[ht!]
  $$[4, 2, 1, 3]$$
  \caption{The compact representation of the Costas array of Figure
  \ref{fig:costas_1}}
  \label{fig:costas_2}
\end{figure}

This representation is equivalent to the definition of a Costas array provided
by Gilbert:

\begin{definition}[Distinct difference permutations]\label{def:costas_1}
  Given a sequence of integers $S$ of length $n$ such that:
    \begin{equation}\label{eq:costas_1}
      0 \leq S_{i} < n
    \end{equation}
  We say that $S$ is a distinct difference permutation $r$ apart if,
  for any given pair $(S_{i}, S_{j})$, satisfies:
    \begin{equation}\label{eq:costas_2}
      S_{i} - S_{i+r} \not \equiv S_{j} - S_{j+r} \bmod n
    \end{equation}
\end{definition}

\begin{definition}[Costas array(Gilbert)]
  Given a sequence $S$ satisfying Equation \ref{eq:costas_1}, we say it's a
  Costas array if, for any given $r$ value, it satisfies Equation
  \ref{eq:costas_2}.
\end{definition}

The compact representation will prove usefull when we introduce the method of
composition to look for new PRN sequences.Several construction methods have
been proposed. For sake of simplicity, we are going to introduce just the Welsh
construction as defined in Gilbert \cite{gilbert_costas}:\\

Given a prime number $p$ and a primitive root $g$ of $p$, we can contruct a
Costas array $S$ as follows:
\begin{equation}
  S_{i} \equiv g^{i} \bmod p
\end{equation}
