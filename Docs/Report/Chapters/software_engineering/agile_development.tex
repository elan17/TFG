\section{Agile development}

    One of the most important tasks to do before starting a project is deciding
    which project management model to follow.\\

    In our case, this project is going to avoid a waterfall model, even though it is the taught model in branches  not focused
    in software engineering. In this case  agile development techniques adapted better to
    our problem for several reasons:
    \begin{itemize}
      \item The client is an active part of the project. This means that
      a lot of feedback can be obtained during development and issues can be fixed earlier in the development cycle.
      \item As this project is being developed in the middle of a health crisis,
      the availability of project resources cannot be predicted. This means
      that having a rigid schedule planned too ahead of time would not  be
      useful.
    \end{itemize}

    \subsection{Role definition}

    Two different roles have been defined:
    \begin{itemize}
      \item Developer, which will design, program, verify and manage the
      software project.
      %In this case, this person corresponds to the author of this report (Juan Toca).
      \item Client, which will provide the requirements for the project, as well
      as validating each iteration.
      % In this case, this person corresponds to  the director of this project (Domingo Pérez).
    \end{itemize}

    \subsection{Iterations}

      In this subsection we will discuss how the development iterations went.\\

      A  convention is that  when we say a C function, we are referring to a Cython
      code without CPython code, in other words, functions that do not call the Python interpreter.

      \subsubsection{Iteration 1: Composite autocorrelation}

      In this iteration, the development was focused on developing an efficient
      way of computing the autocorrelation function of a base sequence with a
      given shift sequence. 3 versions were developed:
      \begin{itemize}
        \item A pure C function that given the autocorrelation of the base
        sequence and the shift sequence computes the autocorrelation.
        \item A wrapper Python function for the previous function which
        computes the autocorrelation of the given base sequence and passes it
        to the C function.
        \item A pure C function which checks if the maximum component of the
        composite autocorrelation exceeds the threshold provided.
      \end{itemize}

      The two first functions are not  part of the actual exhaustive search
      algorithm, but will be useful if properties have to be checked when retrieving the results from the database. \\

      The test process consisted in checking that the python wrapper provided
      the same results as a naive implementation of the algorithm based in
      the convolution theorem. From that, the third function was tested against the
      first one.\\

      At first, C functions with fused types were developed. Although it
      favors extensibility, the compiler presented errors related with
      fused types. Finally  it was decided to drop support for fused types
      as it was only expected to work with integers of 32 bits.\\

      \subsubsection{Iteration 2: Branch and bound algorithm}

      In this iteration we focused on a single threaded C implementation of the
      branch and bound algorithm. This function receives a threshold of the
      maximum autocorrelation the user is interested in, the maximum Hamming
      autocorrelation for the prune part of the algorithm and the base sequence
      to use.\\

      For this purpose, a C implementation of the maximum Hamming
      autocorrelation was developed and an implementation of Legendre sequences
      to be used as base sequences. More types of base sequences can be added
      in the future, but this one was explicitly asked by the client.\\

      The test designed for Legendre sequences exploits its flat
      autocorrelation to check the functions consistency. In the case of
      Hamming's autocorrelation, a test based on lower and upper
      bounds was designed.\\

      In the case of the branch and bound method, it will check that all the
      returned sequences satisfied the specified maximum autocorrelation.\\

      \subsubsection{Iteration 3: Parallelism}

      In this iteration, the main focus was to adapt the branch and bound algorithm
      to a parallel environment. To do that, the algorithm was incrementally
      improved. At first, branch and bound was only applied from a given depth and
      then it was decided to also apply it at the master's process level.\\

      In this case, this iteration was implemented in pure Python as it isn't
      critical code. Most runtime of slave processes will be spent in Cython
      functions and, for simplicity and reliability, it was decided to stick to
      Python.\\

      Tests in this iteration were made in a more manual fashion, running the
      code and checking that all results were coherent. Properly based test were not written in this iteration as the code was not mature enough.\\


      %This was done like this because the code being tested was completly impure and property
      %based tests wouldn't have a worthy coverage to effort relationship to
      %consider writing them.\\

      \subsubsection{Iteration 4: User Interface}

      In this iteration a suitable UI for the program was designed. To do so,
      two functionalities were implemented:
      \begin{itemize}
        \item Support for command line arguments to initialize tasks.
        \item A verbose mode to get statistics of the program.
      \end{itemize}

      Command line parameters complies to POSIX's standard and informs the
      user of possible errors in the input, while the verbose mode logs events
      with its corresponding times to debug the performance of the
      computation.\\

      Again, the testing of this module was purely manual because of all
      the IO involved. The user interface was shown and explained to the client
      to receive their approval.\\
