\section{Agile development}

    One of the most important tasks to do before starting a project is deciding
    which project management model we are going to follow.\\

    In our case, we are going to distance ourself from a waterfall model(even
    though it is the taught model for branches of our degree not focused in
    software engineering) favoring agile development techniques adapted to our
    problem for several reasons:
    \begin{itemize}
      \item The client is an active part of the project. This means we can get
      a lot of feedback during development and fix issues earlier.
      \item As we are developing this project in the middle of a health crisis,
      the availability of project resources cannot be predicted. This mean
      that having a rigid schedule planned too ahead of time wouldn't be
      useful.
    \end{itemize}

    \subsection{Role definition}

    In the development of the project we will define 2 different roles assigned
    to 2 distinct people:
    \begin{itemize}
      \item Developer which will design, program, verify and manage the
      software project. In this case, this person corresponds to the author of
      this report(Juan Toca)
      \item Client which will provide the requirements for the project, as well
      as validating each iteration. In this case, this person corresponds to
      the director of this project(Domingo Pérez)
    \end{itemize}

    \subsection{Iterations}

      In this subsection we will discuss how the development iterations went.\\

      Some conventions: When we say a C function, we are referring to a Cython
      code without CPython code, in other words, functions that doesn't call
      the Python interpreter.

      \subsubsection{Iteration 1: Composite autocorrelation}

      In this iteration, the development was focused in developing an efficient
      way of computing the autocorrelation function of a base sequence with a
      given shift sequence. 3 versions were developed:
      \begin{itemize}
        \item A pure C function that given the autocorrelation of the base
        sequence and the shift sequence computes the autocorrelation
        \item A wrapper Python function for the previous function which
        computes the autocorrelation of the given base sequence and passes it
        to the C function
        \item A pure C function which checks if the maximum component of the
        composite autocorrelation exceeds the threshold provided
      \end{itemize}

      The 2 first functions aren't part of the actual exhaustive search
      algorithm but will be useful if we want to double check it's properties
      when retrieving the results from the database. \\

      The test proccess consisted in checking that the python wrapper provided
      the same results as a naive implementation of the Wiener–Khinchin's
      algorythm. From that, the third function was tested against the first
      one.\\

      At first, we tried to develop C functions with fused types. Although it
      favors extensibility, the compiler started to throw errors related with
      fused types. As the problem would have gone worst when we developed the
      next parts of the program, we decided to drop support for fused types as
      we only expected to work with integers of 32 bits.\\

      \subsubsection{Iteration 2}

      In this iteration we focused on a single threaded C implementation of the
      branch and bound algorythm. This function receives a threshold of the
      maximum autocorrelation we are interested in, the maximum Hamming
      autocorrelation for the bound part of the algorythm and the base sequence
      to use.\\

      For this purpose, we developed a C implementation of the maximum Hamming
      autocorrelation and an implementation of Legendre sequences to be used
      as base sequences(more types of base sequences might be added in the
      future but this one was explicitly asked by the client).\\

      The test designed for Legendre sequences exploits it's flat
      autocorrelation to check the functions consistency. In the case of
      hamming's autocorrelation, we defined a test based on lower and upper
      bounds.\\

      In the case of the branch and bound method, we willl check that all the
      sequences returned satisfies the specified maximum autocorrelation.\\
