\section{Overall description}

  In this chapter we will provide a general idea behind the software project.
  \subsection{Project description}

  This project is developed to assist in the research of binary sequences with
  good autocorrelation functions. This sequences are used in application such
  as radar, wifi or GPS.\\

  The primary focus will be understanding the problem domain to create a
  software as useful as possible. Then, create an extensible program that
  implements an exhaustive search to look for new sequences to add to the
  existing scientific literature.\\

  \subsection{Product functions}

  We will summarize in a general way the functionalities we expect from
  this software.

  \begin{itemize}
    \item Perform computations in search for sequences.\\
            \begin{itemize}
              \item Obtain real-time information of the computation proccess
              \item Change dedicated resources to computation
            \end{itemize}
    \item Manage the set of found sequences
            \begin{itemize}
              \item Query by different parameters of the sequences
              \item Handle found sequences manually
              \item Plot sequence properties
            \end{itemize}
    \item Manage permissions
  \end{itemize}

  \subsection{User classes and characteristics}

  Users of this project can be divided in 2 main groups:\\

  Researchers that will control the search for new sequences, being capable of
  starting computations and modifying the database. This researchers are highly
  specialized personal and are expected to understand common abbreviations in
  the field as well as knowing which parameters are more promising to get a
  successful computation.\\

  System administrators that will optimize the application according to the
  specificiations of the system in which it is deployed. System administrators
  are specialized professional which know how to run benchmarks, maximize
  performance and understand advanced concepts of parallel computing. They are
  expected to be capable of optimizing the software with a manual explaining
  the inner workings of the program in a technical jargon.\\

  In the real world, both roles might overlapp in the same person.\\

  \subsection{Operating Enviroment}

  The computation module is expected to run in highly parallelized systems such
  as super-computers. This is the enviroment in which researchers work so it
  would be a shame if we didn't design the software to take full advantage of
  the capabilities of these systems.\\

  Taking his into account, we should target the software installed in those
  enviroments. This leads us to assume a Linux enviroment(as most supercomputers
  run it) and a Python interpreter(as it's a popular language in the scientific
  community).\\
