\section{Overall description}

  In this section we describe the  general idea behind the software project  and the requirements that needs to fulfill.
  \subsection{Project description}

  The main purpose of this project is to develop a software application to assist
  in the research of binary sequences with good correlation properties. The expected
  end-users are researchers but also designers for wireless communications systems
  such as radar, Wifi or GPS.\\

  The primary challenge is to understand the problem domain to create a software
  as useful as possible ( this first phase has been already explained in the
  previous chapters). Then, it is necessary to create an extensible program that
  assist in the search for new sequences to add to the existing scientific
  literature. Furthermore, a user has to be able to manage the sequences and to
  check and compare their properties. As this process is computationally
  expensive, the software is expected to offer support for parallel computing.\\

 % \subsection{Product functions}

A summary of the functionalities that are expected from
  this software is provided.

  \begin{itemize}
    \item Perform computations in search for new sequences.\\
            \begin{itemize}
              \item Obtain real-time information of the computation process.
              \item Change dedicated resources to computation.
            \end{itemize}
    \item Manage the set of found sequences.
            \begin{itemize}
              \item Handle found sequences manually.
              \item Query by different sequence parameters .
              \item Plot sequence properties.
            \end{itemize}
  \end{itemize}

  \subsection{User classes and characteristics}

  Users of this project can be divided in two main groups:\\

  Researchers that will manage the search for new sequences, being capable of
  starting computations and modifying the database. These researchers are highly
  specialized personnel and are expected to be familiar with common abbreviations in
  the field, as well as knowing which parameters are more promising to get a
  successful computation.\\

  System administrators that will optimize the performance of the software according to the
  specifications of the system in which it is deployed. System administrators
  are specialized professionals which know how to run benchmarks, maximize
  performance and understand advanced concepts of parallel computing.

  %They are  expected to be capable  of optimizing the software with a manual that explains the inner workings of the program in a technical jargon.\\

  In the real world, both roles may overlap in the same person.\\

  \subsection{Operating Environment}

  The computation module is expected to run in highly parallelized systems such
  as super-computers. Most researchers work with this kind of environments, so the software
  would be less competitive if it would not take full advantage of
  the capabilities of these systems.\\

  Taking this into account, the most common software installed in those environments should be
  targeted. This leads us to choose a Linux environment as most supercomputers
  run it.   We have chosen a Python interpreter as it is a popular language in the scientific
  community.\\
