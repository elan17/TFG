\section{Software requirements}

In this section the software requirements of our project will be introduced.  Interviews and document analysis have been the main source to define requirements. Later they have been validated with the intended users.

\subsection{Functional requirements}

The functional requirements of the project are shown in Figure
\ref{functional:fig}.


\begin{figure}[ht!]

  \begin{center}
    \begin{tabular}{||c | p{12cm}||}
      \hline
      Identifier & Description \\
      \hline
      \hline
      FR01 \label{FR01} & The user must be capable to set the base
      sequence to use for the search \\
      \hline
      FR02 \label{FR02} & The user must be capable to set the size
      of the shift sequence to use for the search\\
      \hline
      FR03 \label{FR03} & The user must be capable of setting the maximum
      autocorrelation they are interested in \\
      \hline
       FR04 \label{FR04new} & The user must be capable of store, recover and manage sequences.\\
      \hline
      FR05 \label{FR04} & The user must be capable to extract partial
      results while the computation is still running \\
      %(this doesn't mean that the data must be avalaible with a low latency)
      \hline
      FR06 \label{FR05} & The system administrator must be capable of changing
      the resources assigned to the program\\
      \hline
      FR07 \label{FR06} & The system must provide an interface that shows
      the progress of the computation (No need for low latency
      as it would conflict with \hyperref[NFR01]{NFR01})\\
      \hline
      FR08 \label{FR07} & Software must
      run without supervision  after parameters are established. \\
      \hline
      FR09 \label{FR08} & The parameters of the load balancer must be editable
      by the system administrator in order to adapt to different hardware.
      \\
      \hline
      FR09 \label{FR09} & An system administrator must be capable of managing
      access privileges to the database and running computations\\
      \hline
      FR10 \label{FR10} & The system must provide a way to queue concurrent
      searches \\
      \hline
    \end{tabular}
  \end{center}

  \caption{Functional requierements}
  \label{functional:fig}
\end{figure}

\subsection{Non-functional requirements}

The non-functional requirements of the project are shown in Figure
\ref{non_functional:fig}.

\begin{figure}[ht!]

  \begin{center}

    \begin{tabular}{||c | c | p{7cm} | c||}

    \hline
    Identifier & Type  & Description & Relevance\\
    \hline
    \hline
    NFR01 \label{NFR01} & Performance & As sequences are being searched in a
    huge space, speed is a top priority. &
    Very high\\
    \hline
    NFR02 \label{NFR02} & Compability & As our project is being deployed in a
    environment with different architectures, it should be made as
    compatible as posible. &
    Medium \\
    \hline
    NFR03 \label{NFR03} & Usability & This software is expected to be used by
    specialized personnel. Time spent by users interacting with the application is low. Interface should be minimal. & Very low \\
    \hline
    NFR04 \label{NFR04} & Arquitecture & The program must take full advantage
    of the capabilities of supercomputers, in particular  high parallelization of the system. & High\\
    \hline
    NFR05 \label{NFR05} & Robustness & The program must handle errors such as
   data corruption, miscalculations or precision errors to avoid incorrect results. & Very high \\
    \hline
    NFR06 \label{NFR06} & Robustness & The program cannot have memory leaks that may cause a crash. Restarting memory leaked threads without affecting the
    end result & Medium \\
    \hline
    NFR07 \label{NFR07} & Extensibility & Parts of the software should be reusable to suit each research project needs. & Medium \\
    \hline
    NFR08 \label{NFR08} & Data availability & The availability of the data
    is not  a main concern as the project does not aim to be an interactive
    platform. & Low \\
    \hline
    NFR09 \label{NFR09} & Robustness & The persistence layer must be robust
    enough to avoid data loses since it is costly to produce a result. & High \\
    \hline

    \end{tabular}

  \end{center}

  \caption{Non-functional requirements}
  \label{non_functional:fig}
\end{figure}

\subsection{User interface requirements}

UI design is an important topic of software engineering as the success of a
project is related directly to the users experience and how they relate to the
software.\\

First of all, note that the users are expected to be experienced in the use of
computers, so a complex UI should not be a problem. In this case, even though
the easier the better, our development has a huge constraint on UI design that
should be taken into account: the special type of OS this project
will be deployed in. As the main focus are supercomputers, a minimalist environment
with no graphical desktop is to be expected.\\

For this reason, a command-line interface (CLI) is preferred over other alternatives as it requires fewer system resources, allows scripting and commands can be entered more rapidly as text.
The inferface is divided in two different  main sections:\\
\begin{itemize}
  \item Application launcher (resource allocation, parallelism model, etc.)
  \item Runtime interaction with the system (tasks management, database
  queries, etc.)
\end{itemize}

As most supercomputers run UNIX-based systems, our application should follow
the POSIX\footnote{https://web.archive.org/web/20200804110243/https://pubs.opengroup.org/onlinepubs/9699919799/basedefs/V1_chap12.html} standard on the way it treats arguments.
It should follow conventions such as the use of flags such as --help or
--verbose and providing a man page.\\

It will also provide the capability to store the configuration of the system in case
the application must be restarted quickly (mainly platform specific
configuration such as parameters of the load balancer). \\
