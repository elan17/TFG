\section{Software requeriments}


\subsection{Functional requirements}

\begin{figure}

  \begin{center}
    \begin{tabular}{||c | p{10cm}||}
      \hline
      Identifier & Description \\
      \hline
      \hline
      FR01 \label{FR01} & The researcher must be capable to set the base
      sequence to use for the search \\
      \hline
      FR02 \label{FR02} & The researcher must be capable to set the size
      of the shift sequence to use for the search\\
      \hline
      FR03 \label{FR03} & The researcher must be capable of setting the maximum
      autocorrelation he is interested in \\
      \hline
      FR04 \label{FR04} & The researcher must be capable to extract the
      results while the computation is still running(this doesn't mean that
      the data must be avalaible with a low latency)\\
      \hline
      FR05 \label{FR05} & The system administrator must be capable of changing
      the resources assigned to the program\\
      \hline
      FR06 \label{FR06} & The system must provide an interface that shows
      the progress of the computation(again, there's no need for low latency
      as it would conflict with \hyperref[NFR01]{NFR01})\\
      \hline
    \end{tabular}
  \end{center}

  \caption{Functional requierements}
  \label{functional:fig}
\end{figure}

\subsection{Non-functional requirements}

The functional requirements of the project are shown in Figure
\ref{non_functional:fig}.

\begin{figure}

  \begin{center}

    \begin{tabular}{||c | c | p{7cm} | c||}

    \hline
    Identifier & Type  & Description & Relevance\\
    \hline
    \hline
    NFR01 \label{NFR01} & Performance & As we are searching for sequences in a
    huge space, speed is a top priority to continue with the research &
    Very high\\
    \hline
    NFR02 \label{NFR02} & Compability & As we are deploying our project in a
    research enviroment with different arquitectures, we must try to make it as
    compatible as posible in the case it must be changed from node &
    Medium \\
    \hline
    NFR03 \label{NFR03} & Usability & This software is expected to be used by
    specialized researchers and it isn't an interactive application so the time
    spent dealing with the application by users is low. Interface shouldn't be
    a priority. & Very low \\
    \hline
    NFR04 \label{NFR04} & Arquitecture & The program must take full advantage
    of the capabilities of a supercomputer, in particular the high
    degree of parallelization of the system & High\\
    \hline
    NFR05 \label{NFR05} & Robustness & The program must not produce errors.
    Corruption of data, miscalculations or precision errors must not be
    tolerated as it would screw up the whole result & Very high \\
    \hline
    NFR06 \label{NFR06} & Robustness & The program cannot have memory leaks.
    It's expected to run for a long time and a memory leak can cause a crash.
    Can be fixed by restarting memory leaked threads without affecting the end
    result & Medium \\
    \hline
    NFR07 \label{NFR07} & Extensibility & As the program is used as part of a
    research, we need the parts of the software to be reusable in case the
    research shows a new posible use of the project as part of a new
    development & Medium \\
    \hline
    NFR08 \label{NFR08} & Data availability & The availability of the data
    isn't a main concern as the project doesn't aim to be an interactive
    platform & Low \\
    \hline
    NFR09 \label{NFR09} & Robustness & The persistence layer must be robust
    enough to avoid data loses as it is costly to produce & High \\
    \hline

    \end{tabular}

  \end{center}

  \caption{Non-functional requirements}
  \label{non_functional:fig}
\end{figure}
