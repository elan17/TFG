\section{Software requirements}

In this section we will introduce the software requirements of our project.

\subsection{Functional requirements}

The functional requirements of the project are shown in Figure
\ref{functional:fig}.


\begin{figure}[ht!]

  \begin{center}
    \begin{tabular}{||c | p{12cm}||}
      \hline
      Identifier & Description \\
      \hline
      \hline
      FR01 \label{FR01} & The researcher must be capable to set the base
      sequence to use for the search \\
      \hline
      FR02 \label{FR02} & The researcher must be capable to set the size
      of the shift sequence to use for the search\\
      \hline
      FR03 \label{FR03} & The researcher must be capable of setting the maximum
      autocorrelation he is interested in \\
      \hline
      FR04 \label{FR04} & The researcher must be capable to extract the
      results while the computation is still running (this doesn't mean that
      the data must be avalaible with a low latency)\\
      \hline
      FR05 \label{FR05} & The system administrator must be capable of changing
      the resources assigned to the program\\
      \hline
      FR06 \label{FR06} & The system must provide an interface that shows
      the progress of the computation (again, there's no need for low latency
      as it would conflict with \hyperref[NFR01]{NFR01})\\
      \hline
      FR07 \label{FR07} & Once established the parameters, the software must
      run without needing supervision of any user \\
      \hline
      FR08 \label{FR08} & The parameters of the load balancer must be editable
      by the system admin (as different machines might need different values)
      \\
      \hline
      FR09 \label{FR09} & An administrator must be capable of managing
      privileges for accesing the database and running computations\\
      \hline
      FR10 \label{FR10} & The system must provide a way to queue different
      searches to compute in succesion\\
      \hline
    \end{tabular}
  \end{center}

  \caption{Functional requierements}
  \label{functional:fig}
\end{figure}

\subsection{Non-functional requirements}

The non-functional requirements of the project are shown in Figure
\ref{non_functional:fig}.

\begin{figure}[ht!]

  \begin{center}

    \begin{tabular}{||c | c | p{7cm} | c||}

    \hline
    Identifier & Type  & Description & Relevance\\
    \hline
    \hline
    NFR01 \label{NFR01} & Performance & As we are searching for sequences in a
    huge space, speed is a top priority to continue with the research &
    Very high\\
    \hline
    NFR02 \label{NFR02} & Compability & As we are deploying our project in a
    research enviroment with different arquitectures, we must try to make it as
    compatible as posible in the case it must be changed to another node &
    Medium \\
    \hline
    NFR03 \label{NFR03} & Usability & This software is expected to be used by
    specialized researchers and it isn't an interactive application, so the
    time spent dealing with the application by users is low. Interface
    shouldn't be a priority & Very low \\
    \hline
    NFR04 \label{NFR04} & Arquitecture & The program must take full advantage
    of the capabilities of a supercomputer, in particular the high
    degree of parallelization of the system & High\\
    \hline
    NFR05 \label{NFR05} & Robustness & The program must not produce errors.
    Corruption of data, miscalculations or precision errors must not be
    tolerated as it would screw up the whole result & Very high \\
    \hline
    NFR06 \label{NFR06} & Robustness & The program cannot have memory leaks.
    It's expected to run for a long time and a memory leak can cause a crash.
    It can be fixed by restarting memory leaked threads without affecting the
    end result & Medium \\
    \hline
    NFR07 \label{NFR07} & Extensibility & Since the program is used as part of
    a research, we need the parts of the software to be reusable in case the
    research shows a new posible use for the project as part of a new
    development & Medium \\
    \hline
    NFR08 \label{NFR08} & Data availability & The availability of the data
    isn't a main concern as the project doesn't aim to be an interactive
    platform & Low \\
    \hline
    NFR09 \label{NFR09} & Robustness & The persistence layer must be robust
    enough to avoid data loses since it is costly to produce & High \\
    \hline

    \end{tabular}

  \end{center}

  \caption{Non-functional requirements}
  \label{non_functional:fig}
\end{figure}

\subsection{User interface requirements}

UI design is an important topic of software engineering as the success of a
project is related directly to the users experience and how they relate to the
software.\\

First of all, note that the users are supposed to be experienced in the use of
computers, so a complex UI shouldn't be a problem. In this casse, even though
the easier the better, our development has a huge constraint on UI design that
should be taken into account: the special type of OS we are working with.\\

As we are working with supercomputers, we can encounter a minimalist enviroment
with no graphical desktop. For this reason, we should focus on a command line
based application with 2 different main sections:\\
\begin{itemize}
  \item Application launcher (resource allocation, parallelism model, etc.)
  \item Runtime interaction with the system (tasks management, database
  queries, etc.)
\end{itemize}

As most supercomputers run UNIX-based systems, our application should follow
the POSIX\cite{POSIX_arguments} standard on the way it treats arguments.
It should follow conventions such as the use of flags such as --help or
--verbose and providing a man page.\\

It will also provide a way to store the configuration of the system in case
the application must be restarted quickly (mainly platform specific
configuration such as parameters of the load balancer). \\
