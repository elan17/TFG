\section{Software requirements}

In this section the software requirements of our project will be introduced.  Interviews and document analysis have been the main source to define requirements. Later they have been validated with the intended users.

\subsection{Functional requirements}

The functional requirements of the project are shown in Figure
\ref{functional:fig}.


\begin{figure}[ht!]

  \begin{center}
    \begin{tabular}{||c | p{12cm}||}
      \hline
      Identifier & Description \\
      \hline
      \hline
      FR01 \label{FR01} & The user must be capable to set the base
      sequence to use for the search \\
      \hline
      FR02 \label{FR02} & The user must be capable to set the size
      of the shift sequence to use for the search\\
      \hline
      FR03 \label{FR03} & The user must be capable of setting the maximum
      autocorrelation they are interested in \\
      \hline
       FR04 \label{FR04new} & The user must be capable to store, recover and
       manage sequences.\\
      \hline
      FR05 \label{FR04} & The user must be capable to extract partial
      results while the computation is still running \\
      %(this doesn't mean that the data must be avalaible with a low latency)
      \hline
      FR06 \label{FR05} & The system administrator must be capable of changing
      the resources assigned to the program\\
      \hline
      FR07 \label{FR06} & The system must provide an interface that shows
      the progress of the computation (No need for low latency
      as it would conflict with \hyperref[NFR01]{NFR01})\\
      \hline
      FR08 \label{FR07} & Software must
      run without supervision  after parameters are established. \\
      \hline
      FR09 \label{FR08} & The parameters of the load balancer must be editable
      by the system administrator in order to adapt to different hardware.
      \\
      \hline
      FR09 \label{FR09} & An system administrator must be capable of managing
      access privileges to the database and running computations\\
      \hline
      FR10 \label{FR10} & The system must provide a way to queue concurrent
      searches \\
      \hline
    \end{tabular}
  \end{center}

  \caption{Functional requierements}
  \label{functional:fig}
\end{figure}

\subsection{Non-functional requirements}

The non-functional requirements of the project are shown in Figure
\ref{non_functional:fig}.

\begin{figure}[ht!]

  \begin{center}

    \begin{tabular}{||c | c | p{7cm} | c||}

    \hline
    Identifier & Type  & Description & Relevance\\
    \hline
    \hline
    NFR01 \label{NFR01} & Performance & Speed is a top priority. &
    Very high\\
    \hline
    NFR02 \label{NFR02} & Compability & It should be made as
    compatible as posible with different supercomputers. &
    Medium \\
    \hline
    NFR03 \label{NFR04} & Arquitecture & The program must take full advantage
    of the capabilities of supercomputers, in particular  high parallelization of
    the system. & High\\
    \hline
    NFR04 \label{NFR05} & Robustness & The program must handle errors such as
    data corruption, miscalculations or precision errors to avoid incorrect
    results. & Very high \\
    \hline
    NFR05 \label{NFR06} & Robustness & The program cannot stop unexpectly
    and must provide a way to recover from said crashes & Medium \\
    \hline
    NFR06 \label{NFR07} & Extensibility & Parts of the software should be reusable
    to suit each research project needs. & Medium \\
    \hline
    NFR07 \label{NFR09} & Robustness & The persistence layer must be robust
    enough to avoid data losses. & High \\
    \hline

    \end{tabular}

  \end{center}

  \caption{Non-functional requirements}
  \label{non_functional:fig}
\end{figure}

\subsection{User interface requirements}

UI design is an important topic of software engineering as the sucess of a
project is related directly to the users experience and how they use the
software.\\

First of all, note that the users are expected to be experienced in the use of
computers, so a complex UI should not be a problem. In this case, even though
the easier the better, our development has a huge constraint on UI design that
should be taken into account: the special type of OS this project
will be deployed in. As the main focus are supercomputers, a minimalist environment
with no graphical desktop is preferred.\\

For this reason, a command-line interface (CLI) is preferred over other
alternatives as it requires fewer system resources, allows scripting and commands
can be entered more rapidly as text. The inferface is divided in two different
main sections:\\
\begin{itemize}
  \item Application launcher (resource allocation, parallelism model, etc.)
  \item Runtime interaction with the system (tasks management, database
  queries, etc.)
\end{itemize}

As most supercomputers run UNIX-based systems, our application should follow
the POSIX\footnote{https://web.archive.org/web/20200804110243/https://pubs.opengroup.org/onlinepubs/9699919799/basedefs/V1_chap12.html} standard on the
way it treats arguments.
It should follow conventions such as the use of flags such as --help or
--verbose and providing a man page.\\

It will also provide the capability to store the configuration of the system in
case the application must be restarted quickly (mainly platform specific
configuration such as parameters of the load balancer). \\

The previous discussion were translated to the following implementation of the
command-line interface.
  First, a --help option was developed to list all the flags:
  \begin{lstlisting}
    $ python main.py --help
    usage: python main.py [option...]

    Options and arguments:
    -n : number of threads to use(must be compatible with your MPI enviroment)
         Defaults to the MPI configuration default

    -p : delay between polls in the master thread(higher values will make the
         slaves to wait more until the next task, lower values will increase
         CPU usage of master)

    -s : length of the base sequence(in this version must be a prime number to
         generate Legendre sequences. Other values have undefined behaviour)

    -l : length of shift sequences(this option must be coprime with value
         of s, other values have undefined behaviour)

    -t : size of the task for each thread(this option must be lower than the
         value provided by -l, other values have undefined behaviour)

    -h : maximum hamming autocorrelation allowed(this option must be a
         positive integer other values have undefined behaviour) Defaults to -l

    -c : maximum autocorrelation we are interested in(this option must be a
         positive integer other values have undefined behaviour) Defaults to
         the square root of (-l*-s)

    -v : verbose mode
  \end{lstlisting}

  The verbose mode can log which tasks have been assigned and saves the time stamp
  as well. Finally it tracks the end of child processes and their idle time.

  \begin{lstlisting}
    $ python main.py -s 5 -l 23 -t 20 -c 7 -h 3 -v
    2020-08-23 19:27:49 [1] : TASK_ASSIGNED [0, 0, 1] 9ms
    2020-08-23 19:27:49 [3] : TASK_ASSIGNED [0, 0, 3] 10ms
    2020-08-23 19:27:49 [5] : TASK_ASSIGNED [0, 0, 4] 6ms
    2020-08-23 19:27:49 [7] : TASK_ASSIGNED [0, 1, 1] 17ms
    2020-08-23 19:27:49 [6] : TASK_ASSIGNED [0, 1, 0] 3ms
    2020-08-23 19:27:49 [2] : TASK_ASSIGNED [0, 0, 2] 0ms
                            .
                            .
                            .
    2020-08-23 19:28:41 [1] : TASK_ASSIGNED [0, 4, 0] 0ms
    2020-08-23 19:28:42 [7] : TASK_ASSIGNED [0, 4, 3] 0ms
    2020-08-23 19:28:45 [1] : EXITED
    2020-08-23 19:28:45 [5] : TASK_ASSIGNED [0, 4, 2] 0ms
    2020-08-23 19:28:47 [2] : EXITED
    2020-08-23 19:28:48 [4] : TASK_ASSIGNED [0, 4, 4] 0ms
    2020-08-23 19:28:49 [6] : EXITED
    2020-08-23 19:28:52 [3] : EXITED
    2020-08-23 19:28:55 [5] : EXITED
    2020-08-23 19:28:56 [7] : EXITED
    2020-08-23 19:28:57 [4] : EXITED

  \end{lstlisting}

  Notice that, as standard stdout is used, the log can be piped to a file. The
  output format is designed in a way that eases the use of utilities such as
  awk to process the data of the program (one word message, well defined
  columns, etc.).\\

  Verbosity is completly optional and does not impact performance when inactive.
  Through the experiments, this option is useful to find a good parameter
  configuration.
