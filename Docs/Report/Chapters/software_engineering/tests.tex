\section{Verification}
  One of the most important parts of software development is verifing the
  software. In other words, checking that the semantics of the program built
  are the same as the intended ones. We can waste a lot of time building a
  program to realize that it doesn't work in the last moment and waste a lot
  of time trying to fix it. That fix might generate side effect that break other
  parts of the program and so on. Testing since the early stages of a project
  is mandatory if we want a quality product and an efficient development
  proccess.

  \subsection{Unit tests}

    Unit testing is the smallest piece of test in a project. There exist several
    approaches in the literature such as white box and black box testing. In
    our project, we will take a mixed approach depending on the situation:

    \subsubsection{Property based tests}

    Property-based testing is a not so well known type of black box testing that
    it's built around the idea of defining properties of functions instead of
    test cases. Originally implemented by the Haskell's libray
    "QuickCheck"\cite{QuickCheck}, the paradigm excels at generating huge
    volumes of test cases with just some extra lines of code leading
    to improvements on the coverage over the search space.

    A well implemented library(there are several of them, in our case we are
    working with Hypothesis\cite{Hypothesis} as our project is built with
    Python), should be capable to applying most well practices of black box
    testing such as edge cases, all pairs of edge cases, etc. 
