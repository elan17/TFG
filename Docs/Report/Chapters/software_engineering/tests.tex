\section{Verification}
  One of the most important parts of software development is verifying the
  software. In other words, checking that the semantics of the program built
  are the same as the intended ones. Testing since the early stages of a project
  is mandatory if a quality product and an efficient development
  process is to be accomplish.  A lot of time can be invested building a
  program to realize that it does not work or a fix might generate side effects that break other
  parts of the program and so on.
  \subsection{Unit tests}

    Unit testing is the smallest piece of test suite in a project. There exist
    several approaches in the literature such as white box and black box
    testing. In our project, a mixed approach will be taken depending on the
    situation:

    \subsubsection{Property based tests}

    Property-based testing is a not so well known type of black box testing that
    is built around the idea of defining properties of functions instead of
    test cases. Originally implemented by the Haskell's library
    "QuickCheck"\footnote{
    https://hackage.haskell.org/package/QuickCheck}, this paradigm excels at
    generating huge volumes of test cases with just some extra lines of code leading
    to improvements on the coverage over the search space. It is similar to the
    test automation explained by \citet{Sommerville} in Chapter 23, being the
    main difference that an oracle that predicts the value is not needed.
    Instead, just a property of the output is checked.\\

    A well implemented library(there are several of them, but in our case we are
    working with Hypothesis\footnote{https://hypothesis.readthedocs.io/en/latest/}
    since our project is built in Python) should be capable of applying most well
    practices of black box testing, such as edge cases, all pairs, etc.\\

    The main reason why this type of test suites were chosen is that all the
    properties are already defined in this document and can be used
    straightforward as test cases. In fact, as most of our functions
    are static and pure, the generator will be very simple so the
    tests will take full advantage of this paradigm. In Figure \ref{test_example},
    there is have an example of a property used for testing the codebase.\\

    \begin{figure}[ht!]
      \inputpython{Chapters/software_engineering/test_example.py}{0}{100}
      \caption{An example test for Corollary \ref{autocorrelation:coro:1}}
      \label{test_example}
    \end{figure}

    The problem with this paradigm is that it becomes way too complicated when
    the tested methods have side effects, IO, state machines, etc. As this
    kind of systems usually depend on complex rules to build the generator of
    all the components involved in this systems. For these kind of tests, we
    will rely on the old method of designing test cases by hand.\\

\section{Validation}

The validation proccess in this project depends highly in the iteration in which
we are:

\begin{itemize}
  \item In early iterations, the validation proccess might not be as important
  as the verification proccess. This is because the core functionality of the
  program is an algorithm expressed in a technical manner with little margin
  to misinterpretations.
  \item In later iterations, the validation proccess gains weight in respect of
  the verification proccess as we dive into the UI design. In this case, there
  is more room for missunderstandings between client and developer so we must
  take this proccess into account.
\end{itemize}

Fortunatly, we are working with an agile mindset so a validation
session can be performed often so that the developers introduce the new features to
our client. Then, he can try out the features and point out missunderstandings,
desired changes, etc.\\
