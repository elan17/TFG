\section{Verification}
  One of the most important parts of software development is verifing the
  software. In other words, checking that the semantics of the program built
  are the same as the intended ones. We can waste a lot of time building a
  program to realize that it doesn't work in the last moment and waste a lot
  of time trying to fix it. That fix might generate side effect that break other
  parts of the program and so on. Testing since the early stages of a project
  is mandatory if we want a quality product and an efficient development
  proccess.

  \subsection{Unit tests}

    Unit testing is the smallest piece of test in a project. There exist several
    approaches in the literature such as white box and black box testing. In
    our project, we will take a mixed approach depending on the situation:

    \subsubsection{Property based tests}

    Property-based testing is a not so well known type of black box testing that
    it's built around the idea of defining properties of functions instead of
    test cases. Originally implemented by the Haskell's libray
    "QuickCheck"\cite{QuickCheck}, this paradigm excels at generating huge
    volumes of test cases with just some extra lines of code leading
    to improvements on the coverage over the search space. It's similar to the
    test automation explained by \citet{Sommerville} in Chapter 23, being the
    main difference that we don't need an oracle that predicts the value.
    Instead, we check just a property of the output.\\

    A well implemented library(there are several of them, in our case we are
    working with Hypothesis\cite{Hypothesis} as our project is built in
    Python), should be capable to applying most well practices of black box
    testing such as edge cases, all pairs, etc.\\

    The main reason why we choosed this type of test suites is that all the
    properties are already defined in this document and can be used
    straightforward as test cases. In fact, as most of our functions
    are static and pure, the generator will be very simple so we will take full
    advantage of this paradigm. In Figure \ref{test_example}, we have an example
    of a property used for testing the codebase.\\

    \begin{figure}[ht!]
      \inputpython{Chapters/software_engineering/test_example.py}{0}{100}
      \caption{An example test for Corollary \ref{autocorrelation:coro:1}}
      \label{test_example}
    \end{figure}

    The problem with this paradigm is that it becomes way too complicated when
    we have to test methods with side effects, IO, state machines, etc. As this
    kind of systems usually depend on complex rules to build the generator of
    all the components involved in this systems. For this kind of tests, we
    will rely on the old method of designing test cases by hand.\\
