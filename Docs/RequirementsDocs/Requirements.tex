%Copyright 2014 Jean-Philippe Eisenbarth
%This program is free software: you can
%redistribute it and/or modify it under the terms of the GNU General Public
%License as published by the Free Software Foundation, either version 3 of the
%License, or (at your option) any later version.
%This program is distributed in the hope that it will be useful,but WITHOUT ANY
%WARRANTY; without even the implied warranty of MERCHANTABILITY or FITNESS FOR A
%PARTICULAR PURPOSE. See the GNU General Public License for more details.
%You should have received a copy of the GNU General Public License along with
%this program.  If not, see <http://www.gnu.org/licenses/>.

%Based on the code of Yiannis Lazarides
%http://tex.stackexchange.com/questions/42602/software-requirements-specification-with-latex
%http://tex.stackexchange.com/users/963/yiannis-lazarides
%Also based on the template of Karl E. Wiegers
%http://www.se.rit.edu/~emad/teaching/slides/srs_template_sep14.pdf
%http://karlwiegers.com
\documentclass{scrreprt}
\usepackage{listings}
\usepackage{underscore}
\usepackage[bookmarks=true]{hyperref}
\usepackage[utf8]{inputenc}
\usepackage[english]{babel}
\hypersetup{
    bookmarks=false,    % show bookmarks bar?
    pdftitle={Software Requirement Specification},    % title
    pdfauthor={Jean-Philippe Eisenbarth},                     % author
    pdfsubject={TeX and LaTeX},                        % subject of the document
    pdfkeywords={TeX, LaTeX, graphics, images}, % list of keywords
    colorlinks=true,       % false: boxed links; true: colored links
    linkcolor=blue,       % color of internal links
    citecolor=black,       % color of links to bibliography
    filecolor=black,        % color of file links
    linktoc=page            % only page is linked
}%
\def\myversion{1.0 }
\date{}
urlcolor=purple,        % color of external links
%\title
\usepackage{hyperref}
\begin{document}

\begin{flushright}
    \rule{16cm}{5pt}\vskip1cm
        \Huge{SOFTWARE REQUIREMENTS\\ SPECIFICATION}\\
        \vspace{1.9cm}
        for\\
        \begin{bfseries}
        \vspace{1.9cm}
        Support program for sequence's research\\
        \vspace{1.9cm}
        \LARGE{Version \myversion approved}\\
        \vspace{1.9cm}
        Prepared by Juan Toca Mateo\\
        Universidad de Cantabria\\
        \vspace{1.9cm}
        \today\\
    \end{bfseries}
    \vspace{1.9cm}
\end{flushright}

\tableofcontents

\chapter{Introduction}

\section{Purpose}

This software product aims to make life easier to researchers in the field of
sequences with interesting autocorrelation and crosscorrelation properties.
For that, we will provide a way of computing those sequences and store them for
future use.


\section{Document Conventions}

TODO\\

%$<$Describe any standards or typographical conventions that were followed when
%writing this SRS, such as fonts or highlighting that have special significance.
%For example, state whether priorities  for higher-level requirements are assumed
%to be inherited by detailed requirements, or whether every requirement statement
%is to have its own priority.$>$

\section{Intended Audience}

This document is aimed to each agent involved in the development of this
software project. In one hand, the developer will take it as a guide of the
client needs and, in the other hand, the client will have a document to
check if the final product conforms to the agreement.\\

In particular, the developer is defined as the person that will present it's
bachelor thesis(Juan Toca) and the client is defined as his thesis director
(Domingo Gómez).\\

In case someone wants to build in top of this project, this document provides
the starting point to get a general idea of which issues have been tackled and
in which ways the project could be improved.\\


%$<$Describe the different types of reader that the document is intended for,
%such as developers, project managers, marketing staff, users, testers, and
%documentation writers. Describe what the rest of this SRS contains and how it is
%organized. Suggest a sequence for reading the document, beginning with the
%overview sections and proceeding through the sections that are most pertinent to
%each reader type.$>$

\section{Project Scope}

The main purpose of this project is to provide a graphical framework to assist
in the research of the properties of sequences, more precisely their
autocorrelation and crosscorrelation properties.\\

This software must be capable of exploiting the computational power of a
calculation center to assist in the said research and store in a structured
way the result of this computations.\\

It must be as easy to use as possible, taking into account the characteristics
of it's higly specialized public.\\


\section{References}

TODO\\

%$<$List any other documents or Web addresses to which this SRS refers. These may
%include user interface style guides, contracts, standards, system requirements
%specifications, use case documents, or a vision and scope document. Provide
%enough information so that the reader could access a copy of each reference,
%including title, author, version number, date, and source or location.$>$


\chapter{Overall Description}

\section{Product Perspective}

TODO\\

%$<$Describe the context and origin of the product being specified in this SRS.
%For example, state whether this product is a follow-on member of a product
%family, a replacement for certain existing systems, or a new, self-contained
%product. If the SRS defines a component of a larger system, relate the
%requirements of the larger system to the functionality of this software and
%identify interfaces between the two. A simple diagram that shows the major
%components of the overall system, subsystem interconnections, and external
%interfaces can be helpful.$>$

\section{Product Functions}

We will summarize in a general way the functionalities we expect from
this software.

\begin{itemize}
  \item Perform computations in search for sequences.\\
          \begin{itemize}
            \item Obtain real-time information of the computation proccess
            \item Change dedicated resources to computation
          \end{itemize}
  \item Manage the set of found sequences
          \begin{itemize}
            \item Query by different parameters of the sequences
            \item Handle found sequences manually
            \item Plot sequence properties
          \end{itemize}
  \item Manage permissions
\end{itemize}

\section{User Classes and Characteristics}

Users of this can be divided in 2 main groups:\\

Researchers that will control the search for new sequences, being capable of
starting computations and modifying the database. This researchers are highly
specialized personal and are expected to understand common abbreviations in
the field as well as knowing which parameters are more promising to get a
successful computation.\\

Engineers that will query the database searching for sequences to apply in
a practical use. They are not expected to know about each single detail about
how this sequences are obtained, although they know their properties and how
can be applied in the real world.\\

\section{Operating Environment}

The computation module is expected to run in highly paralelized systems such
as super-computers. This is the enviroment in which researchers work so it
would be a shame if we didn't design the software to take advantage of
that.\\

Taking his into account, we should target the software installed in those
enviroments. This leads us to assume a Linux enviroment(as most supercomputers
run it) and a Python interpreter(as it's a popular language in the scientific
community).\\

%$<$Describe the environment in which the software will operate, including the
%hardware platform, operating system and versions, and any other software
%components or applications with which it must peacefully coexist.$>$

%\section{Design and Implementation Constraints}
%$<$Describe any items or issues that will limit the options available to the
%developers. These might include: corporate or regulatory policies; hardware
%limitations (timing requirements, memory requirements); interfaces to other
%applications; specific technologies, tools, and databases to be used; parallel
%operations; language requirements; communications protocols; security
%considerations; design conventions or programming standards (for example, if the
%customer’s organization will be responsible for maintaining the delivered
%software).$>$

\section{User Documentation}

As this project has a limited scope, we think there is no need in a complete
user manual/wiki appart from a well documented README.md and a man page.\\

Keeping in mind that the end user is supposed to have knowledge in the topic,
contextual help explaining the concepts doesn't need to be too extense.

%$<$List the user documentation components (such as user manuals, on-line help,
%and tutorials) that will be delivered along with the software. Identify any
%known user documentation delivery formats or standards.$>$
\section{Assumptions and Dependencies}

DOES NOT APPLY\\

%$<$List any assumed factors (as opposed to known facts) that could affect the
%requirements stated in the SRS. These could include third-party or commercial
%components that you plan to use, issues around the development or operating
%environment, or constraints. The project could be affected if these assumptions
%are incorrect, are not shared, or change. Also identify any dependencies the
%project has on external factors, such as software components that you intend to
%reuse from another project, unless they are already documented elsewhere (for
%example, in the vision and scope document or the project plan).$>$


\chapter{External Interface Requirements}

\section{User Interfaces}

$<$Describe the logical characteristics of each interface between the software
product and the users. This may include sample screen images, any GUI standards
or product family style guides that are to be followed, screen layout
constraints, standard buttons and functions (e.g., help) that will appear on
every screen, keyboard shortcuts, error message display standards, and so on.
Define the software components for which a user interface is needed. Details of
the user interface design should be documented in a separate user interface
specification.$>$

\section{Hardware Interfaces}

Taking into account that the software module in charge of running the
computations might be deployed in a cluster, we must consider their
characteristics. Their architecture of highly paralelized systems rely on
a problematic choke point, the interconnection network.\\

Each time a proccess needs to read/write data in a memory address shared with
a proccess in other node, this information must be sent through the
interconnection network. This will slow down significantly the performance of
the nodes involved.We must take this hardware constraint into account when
choosing the parallel programming model.\\

%$<$Describe the logical and physical characteristics of each interface between
%the software product and the hardware components of the system. This may include
%the supported device types, the nature of the data and control interactions
%between the software and the hardware, and communication protocols to be
%used.$>$

\section{Software Interfaces}
$<$Describe the connections between this product and other specific software
components (name and version), including databases, operating systems, tools,
libraries, and integrated commercial components. Identify the data items or
messages coming into the system and going out and describe the purpose of each.
Describe the services needed and the nature of communications. Refer to
documents that describe detailed application programming interface protocols.
Identify data that will be shared across software components. If the data
sharing mechanism must be implemented in a specific way (for example, use of a
global data area in a multitasking operating system), specify this as an
implementation constraint.$>$

\section{Communications Interfaces}

$<$Describe the requirements associated with any communications functions
required by this product, including e-mail, web browser, network server
communications protocols, electronic forms, and so on. Define any pertinent
message formatting. Identify any communication standards that will be used, such
as FTP or HTTP. Specify any communication security or encryption issues, data
transfer rates, and synchronization mechanisms.$>$


\chapter{System Features}
$<$This template illustrates organizing the functional requirements for the
product by system features, the major services provided by the product. You may
prefer to organize this section by use case, mode of operation, user class,
object class, functional hierarchy, or combinations of these, whatever makes the
most logical sense for your product.$>$

\section{System Feature 1}
$<$Don’t really say “System Feature 1.” State the feature name in just a few
words.$>$

\subsection{Description and Priority}
$<$Provide a short description of the feature and indicate whether it is of
High, Medium, or Low priority. You could also include specific priority
component ratings, such as benefit, penalty, cost, and risk (each rated on a
relative scale from a low of 1 to a high of 9).$>$

\subsection{Stimulus/Response Sequences}
$<$List the sequences of user actions and system responses that stimulate the
behavior defined for this feature. These will correspond to the dialog elements
associated with use cases.$>$

\subsection{Functional Requirements}
$<$Itemize the detailed functional requirements associated with this feature.
These are the software capabilities that must be present in order for the user
to carry out the services provided by the feature, or to execute the use case.
Include how the product should respond to anticipated error conditions or
invalid inputs. Requirements should be concise, complete, unambiguous,
verifiable, and necessary. Use “TBD” as a placeholder to indicate when necessary
information is not yet available.$>$

$<$Each requirement should be uniquely identified with a sequence number or a
meaningful tag of some kind.$>$

REQ-1:	REQ-2:

\section{System Feature 2 (and so on)}


\chapter{Other Nonfunctional Requirements}

\section{Performance Requirements}
$<$If there are performance requirements for the product under various
circumstances, state them here and explain their rationale, to help the
developers understand the intent and make suitable design choices. Specify the
timing relationships for real time systems. Make such requirements as specific
as possible. You may need to state performance requirements for individual
functional requirements or features.$>$

\section{Safety Requirements}
$<$Specify those requirements that are concerned with possible loss, damage, or
harm that could result from the use of the product. Define any safeguards or
actions that must be taken, as well as actions that must be prevented. Refer to
any external policies or regulations that state safety issues that affect the
product’s design or use. Define any safety certifications that must be
satisfied.$>$

\section{Security Requirements}
$<$Specify any requirements regarding security or privacy issues surrounding use
of the product or protection of the data used or created by the product. Define
any user identity authentication requirements. Refer to any external policies or
regulations containing security issues that affect the product. Define any
security or privacy certifications that must be satisfied.$>$

\section{Software Quality Attributes}
$<$Specify any additional quality characteristics for the product that will be
important to either the customers or the developers. Some to consider are:
adaptability, availability, correctness, flexibility, interoperability,
maintainability, portability, reliability, reusability, robustness, testability,
and usability. Write these to be specific, quantitative, and verifiable when
possible. At the least, clarify the relative preferences for various attributes,
such as ease of use over ease of learning.$>$

\section{Business Rules}
$<$List any operating principles about the product, such as which individuals or
roles can perform which functions under specific circumstances. These are not
functional requirements in themselves, but they may imply certain functional
requirements to enforce the rules.$>$


\chapter{Other Requirements}
$<$Define any other requirements not covered elsewhere in the SRS. This might
include database requirements, internationalization requirements, legal
requirements, reuse objectives for the project, and so on. Add any new sections
that are pertinent to the project.$>$

\section{Appendix A: Glossary}
%see https://en.wikibooks.org/wiki/LaTeX/Glossary
$<$Define all the terms necessary to properly interpret the SRS, including
acronyms and abbreviations. You may wish to build a separate glossary that spans
multiple projects or the entire organization, and just include terms specific to
a single project in each SRS.$>$

\section{Appendix B: Analysis Models}
$<$Optionally, include any pertinent analysis models, such as data flow
diagrams, class diagrams, state-transition diagrams, or entity-relationship
diagrams.$>$

\section{Appendix C: To Be Determined List}
$<$Collect a numbered list of the TBD (to be determined) references that remain
in the SRS so they can be tracked to closure.$>$

\end{document}
